\documentclass[12pt]{article}
\usepackage{zed-csp}%from
\usepackage[utf8]{inputenc}

\begin{document}

\title{Schémas Z et tests BlackBox}
\author{Groupe 2}
\date{\today}
\maketitle


\section{Types donnés}
Les spécifications Z s'appuient sur les types suivants:
\begin{zed}
AMBULANCE \\
COORDINATES \\
INCIDENT \\
NAME \\
DATE \\
STATUS ::= 'FREE' | 'CHOSEN' | 'MOBILIZED' \\
KIND ::= 'NORMAL' | 'MEDICALIZED' \\
LOCALISATION \\
BOOLEAN ::= 'TRUE'  | 'FALSE' \\
MESSAGE ::= \\
\end{zed}

% Expliciter chacun des types ci-après en expliquant ce que chacun représente
L'explication pour chacun des types est fournie dans le tableau \ref{tab:typedefinition}.
\begin{table}[htdp]
\begin{center}\begin{tabular}{|c|c|}\hline 
\textit{Type} & \textit{Définition} \\\hline
AMBULANCE & null \\\hline COORDINATES & null \\\hline INCIDENT & null \\\hline NAME & null \\\hline DATE & null \\\hline STATUS & null \\\hline KIND & null \\\hline LOCALISATION & null \\\hline BOOLEAN & null \\\hline MESSAGE & null \\\hline \end{tabular} \caption{Définition des types donnés}
\end{center}
\label{tab:typedefinition}
\end{table}


\section{Schémas Types}
\begin{schema}{AmbulanceInformation}
 registered\_ambulance : \power AMBULANCE \\
 position\_ambulance : AMBULANCE \pinj COORDINATES \\
 kind\_ambulance : AMBULANCE \pfun KIND \\
  
  \where
  
  \dom position\_ambulance = \dom kind\_ambulance = registered\_ambulance
  
\end{schema}

\begin{schema}{IncidentInformation}
  \where
\end{schema}

\begin{schema}{Mobilization}
  \where
\end{schema}

\section{Opérations}

\begin{schema}{ChooseBestAmbulanceOK}
  \where
\end{schema}

\begin{schema}{ListUnattributedIncidents}
  \where
\end{schema}

\section{Tests Boîte Noire}




\section{Exemples Z}
Cette section est laissée ici pour faciliter le démarrage de tout le monde avec Z en latex.  Évidemment\\ elle devra être retirée du rapport final.

Exemples de types donnés:
\begin{zed}
[PRODUCT\\ COIN]
\end{zed}


Exemple de schéma d'état
\begin{schema}{ShemaName}
  inventory : \bag PRODUCT\\
  price : PRODUCT \pfun \nat\\
  float : \bag COIN\\
  entered : \bag COIN\\
  accept : \power COIN
  \where
  \dom inventory \subseteq \dom price\\
  \dom float \subseteq accepted\\
  \dom entered \subseteq accepted
\end{schema}

Exemple d'opération
\begin{schema}{AcceptCoin}
\Delta Money\\
newCoin? : COIN
\where
newCoin? \notin accept\\
float' = float\\
entered' = entered\\
accept' = accept \cup \{newCoin?\}
\end{schema}


\end{document}
