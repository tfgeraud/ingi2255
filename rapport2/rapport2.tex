\documentclass{report}

\usepackage[utf8]{inputenc}
\usepackage{style-report}

% L'en-tête et la mise en page sera différente
% Merci de ne pas faire de mise en page dans le document !
\title{London Ambulance System}
\subtitle{Schémas Z et tests BlackBox}
\author{\normalsize{Groupe 2}\\
\footnotesize{
Simon Busard, \\
Antoine Cailliau, \\
Laurent Champon,\\
Erick Lavoie, \\
Quentin Pirmez,\\
Frederic Van der Essen, \\
Géraud Talla Fotsing}}

\date{\today}

% Pour modifier la géométrie de la page
\usepackage{geometry}
\geometry{a4paper,left=3cm,right=3cm,top=2.5cm,bottom=4cm}

\newlength{\realtextwidth}
\setlength{\realtextwidth}{\textwidth}
\newcommand{\insertfiguremargin}[2]{
	\begin{figure}[!h]
	\noindent\begin{minipage}[!h]{\marginparsep+\marginparwidth+\marginparpush+\textwidth}
		\begin{minipage}[t]{\realtextwidth}
			\vspace{0pt}
			\includegraphics[width=\realtextwidth]{images/#1.png}
		\end{minipage}
		\hspace{\marginparsep}
		\begin{minipage}[t]{\marginparwidth+\marginparpush}
			\vspace{0pt}
			\setcaptionwidth{\marginparwidth+\marginparpush}
			\caption{#2}\label{fig:#1}
			\setcaptionwidth{0.9\realtextwidth}
		\end{minipage}		
	\end{minipage}
	\end{figure}

}

\setlength{\realtextwidth}{\textwidth}
\newcommand{\insertfigure}[2]{
	\begin{figure}[!h]
	\noindent\begin{minipage}[!h]{\marginparsep+\marginparwidth+\marginparpush+\textwidth}
		\begin{minipage}[t]{\textwidth}
			\vspace{0pt}
			\includegraphics[width=\textwidth]{images/#1.png}
		\end{minipage}
		
		\hspace{\realtextwidth}
		\hspace{\marginparsep}
		\begin{minipage}[t]{\marginparwidth+\marginparpush}
			\vspace{0.5cm}
			\setcaptionwidth{\marginparwidth+\marginparpush}
			\caption{#2}\label{fig:#1}
			\setcaptionwidth{0.9\realtextwidth}
		\end{minipage}		
	\end{minipage}
	\end{figure}

}

\usepackage{zed-csp}%from

\begin{document}

\setlength{\parskip}{1em}
\startdocument

\maketitle
\setcounter{tocdepth}{1}
\tableofcontents

\chapter{Schémas Z}

Dans cette section, nous présentons les différentes opérations et
schémas type liés que nous avons formalisés à l'aide du language Z.

Plus précisément, la section \ref{sec:type} présente les différents
types sur lequels s'appuie le reste de la formalisation. La section
\ref{sec:sctype} présente les schémas types. Enfin la section
\ref{sec:op} présente les opérations formalisées.

\section{Types donnés}\label{sec:type}
Les spécifications Z de nos schémas types et 
opérations s'appuient sur les types présentés ci-dessous. Ces types sont
définis par la suite.
\begin{zed}
AMBULANCE \\
COORDINATES \\
INCIDENT \\
STATUS  \\
KIND \\
LOCALISATION \\
BOOLEAN \\
REPORT  \\
DESCRIPTION
\end{zed}

Pour certains types, il est intéressant de spécifier les valeurs 
possibles de ces types. Ces différentes valeurs possibles sont présentées
ci-dessous.
% FIXME: Check if quotes (') are needed around type... I'm not sure.
\begin{syntax}
STATUS ::= FREE | CHOSEN | MOBILIZED \\
KIND ::= NORMAL | MEDICALIZED \\
BOOLEAN ::= TRUE  | FALSE \\
MESSAGE ::= OK | UnknownIncident | IncidentAlreadyAttributed | NoGoodAmbulance
\end{syntax}

Les types une fois déclarés doivent évidemment être définis de 
manière précise et claire. 
\begin{table}[!h]
\begin{minipage}{\marginparsep+\marginparwidth+\marginparpush+\textwidth}
	\begin{tabularx}{\textwidth}{|l|X|}
		\hline 
		\textit{Type} & \textit{Définition} \\
		\hline
		AMBULANCE & L'ensemble des ambulances physiques possibles. \\
		\hline
		COORDINATES & L'ensemble des coordonnées géographiques possibles (latitude, longitude). \\
		\hline
		INCIDENT & L'ensemble des incidents possibles. \\
		\hline
		STATUS & L'ensemble des statuts possibles pour une ambulance. \\
		\hline
		KIND & L'ensemble des types d'ambulance possibles. \\
		\hline
		LOCALISATION & L'ensemble des localisations possibles (rue, numéro, adresse, etc.). \\
		\hline
		BOOLEAN & L'ensemble des booléens. \\
		\hline
		REPORT & L'ensemble des messages possibles. \\
		\hline
		DESCRIPTION & L'ensemble des descriptions d'incident possibles. \\
		\hline
	\end{tabularx}
	\caption{Définition des types donnés}\label{tab:typedefinition}
\end{minipage}
\end{table}

\section{Schémas Types}\label{sec:sctype}
Cette section présente les différents schémas types nécessaires pour la 
bonne formalisation de nos états. Les trois schémas types sont 
les suivants: 
\begin{itemize}
	\item AmbulanceInformation
	\item IncidentInformation
	\item Mobilization
\end{itemize}

\begin{schema}{AmbulanceInformation}
 	registered\_ambulance : \power AMBULANCE \\
	\newline \\
 	position\_ambulance : AMBULANCE \pinj COORDINATES \\
 	kind\_ambulance : AMBULANCE \pfun KIND \\
  \where
  	\dom position\_ambulance = \dom kind\_ambulance = registered\_ambulance
\end{schema}

\begin{schema}{IncidentInformation}
	registered\_incident : \power INCIDENT \\
	\newline \\
	victimAge\_incident : INCIDENT \pfun \nat \\
	victimPregnant\_incident : INCIDENT \pfun BOOLEAN \\
	localisation\_incident : INCIDENT \pfun LOCALISATION \\
	description\_incident : INCIDENT \pfun DESCRIPTION \\
	\newline \\
	position\_incident : INCIDENT \pfun COORDINATES \\
	ambulanceKindNeeded\_incident : INCIDENT \pfun KIND \\
  \where
  	\dom victimAge\_incident = \dom victimPregnant\_incident =\\
	 \dom localisation\_incident = \dom description\_incident =\\
	 registered\_incident \\
	 \newline \\
	 \dom position\_incident = \dom ambulanceKindNeeded\_incident \\
	 \dom position\_incident \subseteq registered\_incident \\
	 \dom ambulanceKindNeeded\_incident  \subseteq registered\_incident \\ % Redondant répété pour la clarté
\end{schema}

La dernière ligne du schéma \textit{IncidentInformation} est redondante
mais nous avons préféré l'expliciter pour plus de clarté.

\begin{schema}{Mobilization}
	AmbulanceInformation \\
	IncidentInformation \\
	\newline \\
	status\_ambulance : AMBULANCE \pfun STATUS \\
	choice\_mobilization : AMBULANCE \pinj INCIDENT \\
	mob\_mobilization : AMBULANCE \pinj INCIDENT \\
  \where
  	\dom choice\_mobilization \subseteq registered\_ambulance \\
	\dom mob\_mobilization \subseteq registered\_ambulance \\
	\dom mob\_mobilization  \subseteq \dom choice\_mobilization \\
	\newline \\
	\ran choice\_mobilization \subseteq registered\_incident \\
	\ran mob\_mobilization \subseteq registered\_incident \\
	\ran mob\_mobilization  \subseteq \ran choice\_mobilization \\
	\newline \\
	\forall a: AMBULANCE @ 
			(status\_ambulance(a) = FREE) \equiv \\
	                        ( a \notin \dom choice\_mobilization \land a \notin \dom mob\_mobilization) \\
	                   \land
	                   (status\_ambulance(a) = CHOSEN) \equiv \\
	                        ( a \in \dom choice\_mobilization \land a \notin \dom mob\_mobilization) \\
	                     \land
	                   (status\_ambulance(a) = MOBILIZED) \equiv \\
	                        ( a \in \dom choice\_mobilization \land a \in \dom mob\_mobilization) \\
  	
\end{schema}

\section{Opérations}\label{sec:op}

Cette section présente deux opérations formalisées à l'aide du langage Z:
ChooseBestAmbulance et ListUnattributedIncidents.

\section{Le choix de l'ambulance : ChooseBestAmbulance}

L'opération ChooseBestAmbulance est, pour rappel, l'opérationalisation
du but \textit{Achive[BestAmbulanceChosen]} contribuant directemment à l'affection
d'une ambulance sur un incident. Cette opération
peut être séparée comme suit :
\begin{eqnarray*}
ChooseBestAmbulance & = & (ChooseBestAmbulanceOK \land Success) \\ && \lor
 UnknownIncident \\ && \lor IncidentAlreadyAttributed \\ && \lor NoGoodAmbulance
\end{eqnarray*}
Avec une telle décomposition, l'opération est robuste. La suite détaille 
naturellement les opérations de cette décomposition.

\subsection{ChooseBestAmbulanceOK}

Pour cette première opération, nous faisons comme hypothèse qu'il existe une
fonction $$distance : (COORDINATES, COORDINATES) \rightarrow \nat$$ qui retourne la 
distance entre deux coordonnées.

\begin{schema}{ChooseBestAmbulanceOK}
	\Delta Mobilization 	\\
	\Xi AmbulanceInformation\\
	\Xi IncidentInformation	\\
	\newline				\\
	incident? : INCIDENT	\\
	\newline				\\
	ambulance! : AMBULANCE	\\
  \where
  	incident? \in registered\_incident \\
  	incident? \notin \ran choice\_mobilization \\
  	\exists a: AMBULANCE @ status\_ambulance = FREE \land \\
  	 kind\_ambulance(a) = ambulanceKindNeeded\_incident(incident?) \\
  	\newline \\
  	\forall a: AMBULANCE @ (status\_ambulance(a) = FREE \land \\
  	 kind\_ambulance(a) = ambulanceKindNeeded\_incident(incident?) \land \\
  	 distance(position\_ambulance(a), position\_incident(incident?)) \\
  	 \geq distance(position\_ambulance(ambulance!), \\
  	 position\_incident(incident?)))\\
  	\newline \\
  	choice\_mobilization' = choice\_mobilization \oplus \{ ambulance! \mapsto incident? \} \\
  	mob\_mobilization' = mob\_mobilization
\end{schema}

	Nous n'explicitons pas ici le nouvel état de $status\_ambulance$ car il
	est mis à jour automatiquement par le dernier invariant de
	\textit{Mobilization}.

\subsection{Success}

\begin{schema}{Success}
	message! : REPORT	\\
  \where				
	message! = OK		\\
\end{schema}

\subsection{UnknownIncident}

\begin{schema}{UnknownIncident}
	\Xi IncidentInformation	\\
	\newline				\\
	incident? : INCIDENT	\\
	\newline				\\
	ambulance! : AMBULANCE	\\
	message! : REPORT		\\
  \where  
	incident? \notin registered\_incident \\
	message! = UnknownIncident
\end{schema}

\subsection{IncidentAlreadyAttributed}

\begin{schema}{IncidentAlreadyAttributed}
	\Xi IncidentInformation	\\
	\Xi Mobilization		\\
	\newline				\\
	incident? : INCIDENT	\\
	\newline				\\
	ambulance! : AMBULANCE	\\
	message! : REPORT		\\
  \where
  	incident? \in registered\_incident \\
  	incident? \in \ran choice\_mobilization \\
  	\newline \\
  	message! = IncidentAlreadyAttributed
\end{schema}
\newpage
\subsection{NoGoodAmbulance}

\begin{schema}{NoGoodAmbulance}
	\Xi Mobilization		\\
	\newline				\\
	incident? : INCIDENT	\\
	\newline				\\
	ambulance! : AMBULANCE	\\
	message! : REPORT		\\
  \where
	incident? \in registered\_incident \\
	incident? \notin \ran choice\_mobilization \\
	\not \exists a : AMBULANCE @ status\_ambulance(a) = FREE \land \\
	 kind\_ambulance(a) = ambulanceKindNeeded\_incident(incident?) \\
	\newline \\
	message! = NoGoodAmbulance
\end{schema}

\section{Liste des ambulances non-attribuées : ListUnattributedIncidents}
Cette opération est une simple requête sur l'état de notre systême. Cette
opération retourne l'ensemble des incidents pour lesquels aucune ambulance
n'a encore été choisie.

Nous pouvons définir l'opération comme suivant
\begin{eqnarray*}
ListUnattributedIncidents & = & (ListUnattributedIncidentsOK \land Success)
\end{eqnarray*}

\begin{schema}{ListUnattributedIncidentsOK}
	\Xi Mobilization		\\
	\newline				\\
	incidents! : \power INCIDENT \\
  \where
	incidents! \subseteq registered\_incident \\
	incidents! = \{ i : INCIDENT | i \notin \ran choice\_mobilization \land i \notin \ran mob\_mobilization \}
\end{schema}

\chapter{Tests Boîte Noire}

Ce chapitre présente les tests boîtes noires qui sont dérivés directement
des opérations présentées ci-dessus.

\section{ChooseBestAmbulance}

Pour le test en boîte noire, il nous faut présenter les entrées,
les sorties, la table de vérité liée à ces entrées et sorties et enfin, 
des représentants que nous utiliserons dans les tests concrets.

\subsection{Inputs}

Premièrement, définissons les préconditions sur les entrées de l'opération:

\noindent\begin{minipage}{\marginparsep+\marginparwidth+\marginparpush+\textwidth}
\begin{tabularx}{\textwidth}{|l|X|}
	\hline
	I1 & $incident? \in registered\_incident$ \\
	\hline
	I2 & $incident? \notin \ran choice\_incident$ \\
	\hline
	I3 & $\exists a: AMBULANCE @ status\_ambulance = FREE$\\&$ \land
  	 kind\_ambulance(a) = ambulanceKindNeeded\_incident(incident?)$ \\
	\hline
\end{tabularx}
\end{minipage}

\subsection{Outputs}

Deuxièmement, définissons les sorties possibles.

\noindent\begin{minipage}{\marginparsep+\marginparwidth+\marginparpush+\textwidth}
\begin{tabularx}{\textwidth}{|l|X|}
	\hline
	O1 & $\forall a: AMBULANCE @ (status\_ambulance(a) = FREE$\\&$\land
  	 kind\_ambulance(a) = ambulanceKindNeeded\_incident(incident?)$\\&$\land
  	 distance(position\_ambulance(a), position\_incident(incident?)))$\\&$
  	 \geq distance(position\_ambulance(ambulance!), position\_incident(incident?))$ \\
  	\hline
  	O2 & $choice\_mobilization' = choice\_mobilization \oplus \{ ambulance! \mapsto incident? \}$ \\
  	\hline
  	O3 & $mob\_mobilization' = mob\_mobilization$ \\
  	\hline
  	O4 & $message! = \{OK, UI, IAA, NGA\}$ \\
	\hline
	O5 & $status\_mobilization' = status\_mobilization \oplus \{ ambulance! \mapsto CHOSEN \}$ \\
  	\hline
\end{tabularx}
\end{minipage}

Notons par ailleurs que $UI$ équivant à $UnknownIncident$, $IAA$ à $IncidentAlreadyAttributed$ et $NGA$ à $NoGoodAmbulance$.

\subsection{Black-box decision table}

Troisièmement, énonçons la table de vérité, liant les deux précédentes
étapes.

\begin{tabular}{|c|c|c|c|c|}
	\hline
	I1 & 1 & 0 & 1 & 1 \\
	I2 & 1 & - & 0 & 1 \\
	I3 & 1 & - & - & 0 \\
	\hline \hline
	O1 & X &   &   &   \\
	O2 & X &   &   &   \\
	O3 & X & X & X & X \\
	O5 & X &   &   &   \\
	O4 & $OK$ & $UI$ & $IAA$ & $NGA$ \\
	\hline
\end{tabular}

\subsection{Cas de test concrets}

Enfin, nous devons préparer des tests concrets sur chacune des classes 
présentées dans la table de vérité (une classe étant équivalente à une
colonne).

Supposons que le test s'exécute sur l'ensemble de départ suivant:
\begin{eqnarray*}
registered\_ambulance &=& \{alpha1, alpha2, alpha3, mike1, mike2\}\\
position\_ambulance &=& \{alpha1 \rightarrow (0,0), alpha2 \rightarrow (0,0), alpha3 \rightarrow (0,0), \\ 
	&& mike1 \rightarrow (0,0), mike2 \rightarrow (0,0)\}\\
kind\_ambulance &=& \{alpha1 \rightarrow NORMAL, alpha2 \rightarrow NORMAL, alpha3 \rightarrow NORMAL, \\ 
	&& mike1 \rightarrow MEDICALIZED, mike2 \rightarrow MEDICALIZED\} \\
\\
registered\_incident &=& \{incident1, incident2, incident3, incident4\}\\
victimPregnant\_incident &=& \{incident1 \rightarrow FALSE, incident2\rightarrow FALSE, \\ 
	&& incident3\rightarrow FALSE, incident4\rightarrow FALSE\}\\
localisation\_incident &=& \{incident1 \rightarrow \text{"Rue Léonard, 7"}, incident2 \rightarrow \text{"Allée du Dr. Cooper"},\\ 
	&&  incident3  \rightarrow \text{"Penny's Street"}, incident4 \rightarrow \text{"Wolowitz's lane"}\}\\
description\_incident &=& \{incident1 \rightarrow \text{"Left arm broken"}, incident2 \rightarrow \text{"Right leg broken"},\\ 
	&&  incident3 \rightarrow \text{"Coma suspected"}, incident4 \rightarrow \text{"Injure at head"}\}\\
position\_incident &=& \{incident1 \rightarrow (1,0), incident2 \rightarrow (1,1),\\ 
	&&  incident3 \rightarrow (3,5), incident4 \rightarrow (1,3)\}\\
ambulanceKindNeeded\_incident &=& \{incident1 \rightarrow NORMAL, incident2 \rightarrow NORMAL, \\ 
	&& incident3 \rightarrow MEDICALIZED, incident4 \rightarrow NORMAL\}\\
\\
status\_ambulance &=& \{alpha1 \rightarrow FREE, alpha2 \rightarrow CHOSEN, alpha3 \rightarrow FREE, \\ 
	&& mike1 \rightarrow MOBILIZED, mike2 \rightarrow FREE\} \\
choice\_ambulance &=& \{alpha2 \rightarrow incident1\} \\
mob\_mobilization &=& \{alpha3 \rightarrow incident4, mike1 \rightarrow incident3\} \\
\end{eqnarray*}

\subsubsection{Première classe}

En entrée, nous avons $incident2$ et en sortie, 
\begin{itemize}
	\item $choice\_ambulance' = \{alpha2 \rightarrow incident1, alpha1 \rightarrow incident3 \}$
	\item $mob\_mobilization' = mob\_mobilization$
	\item $status\_ambulance' = status\_ambulance \oplus \{alpha2 \rightarrow CHOSEN \}$
	\item $message! = OK$
\end{itemize}

\subsubsection{Seconde classe}

En entrée, nous avons $incident5$ et en sortie, 
\begin{itemize}
	\item $choice\_ambulance' = choice\_ambulance$
	\item $mob\_mobilization' = mob\_mobilization$
	\item $status\_ambulance' = status\_ambulance$
	\item $message! = UnknownIncident$
\end{itemize}

\subsubsection{Troisième classe}

En entrée, nous avons $incident1$ et en sortie, 
\begin{itemize}
	\item $choice\_ambulance' = choice\_ambulance$
	\item $mob\_mobilization' = mob\_mobilization$
	\item $status\_ambulance' = status\_ambulance$
	\item $message! = IncidentAlreadyAttributed$
\end{itemize}

\subsubsection{Quatrième classe}

Pour ce test, il nous faut modifier l'ensemble de départ.
\begin{eqnarray*}
status\_ambulance &=& \{alpha1 \rightarrow MOBILIZED, alpha2 \rightarrow CHOSEN, alpha3 \rightarrow FREE, \\ 
	&& mike1 \rightarrow MOBILIZED, mike2 \rightarrow FREE\} \\
mob\_mobilization &=& \{alpha3 \rightarrow incident4, alpha1 \rightarrow incident2, mike1 \rightarrow incident3\} \\
\end{eqnarray*}
En entrée, nous avons $incident2$ et en sortie, 
\begin{itemize}
	\item $choice\_ambulance' = choice\_ambulance$
	\item $mob\_mobilization' = mob\_mobilization$
	\item $status\_ambulance' = status\_ambulance$
	\item $message! = NoGoodAmbulance$
\end{itemize}

Un autre test à exécuter serait si aucune ambulance n'existe dans 
le système, il est alors possible reprendre tous les tests
précédents qui devraient renvoyer le message $NoGoodAmbulance$ pour tout
les tests dans modifier $choice\_ambulance$, $mob\_mobilization$ ou 
$status\_ambulance$.

\section{ListUnattributedIncidents}

Comme pour le test précédent, il nous faut présenter les entrées,
les sorties, la table de vérité liée à ces entrées et sorties et enfin, 
des représentants que nous utiliserons dans les tests concrets.

\subsection{Inputs}

La précondition étant vide, nous n'avons pas de conditions à énoncer ici.

\subsection{Outputs}

\noindent\begin{minipage}{\marginparsep+\marginparwidth+\marginparpush+\textwidth}
\begin{tabularx}{\textwidth}{|l|X|}
	\hline
	O1 & $incidents! \subseteq registered\_incident$ \\ \hline
	O2 & $incidents! = \{ i : INCIDENT | i \notin \ran choice\_mobilization \land i \notin \ran mob\_mobilization \}$ \\ \hline
	O3 & $message! = \{OK\}$ \\ \hline
  	\hline
\end{tabularx}
\end{minipage}

\subsection{Black-box decision table}

\begin{tabular}{|c|c|}
	\hline
	O1 & X \\
	O2 & X \\
	O3 & $OK$ \\
	\hline
\end{tabular}

\subsection{Cas de test concrets}

\subsubsection{Premier test}
Si nous reprenons notre ensemble de valeurs de départ utilisé pour l'opération
précédente, la sortie attendue de notre opération sera la suivante:
\begin{itemize}
	\item $incidents! = \{incident2\}$
\end{itemize}

Nous pouvons proposer plusieurs autres tests basés sur différentes
versions de l'ensemble de départ.

\subsubsection{Second test}
Tous les incidents n'ont pas d'ambulance affectée. 
\begin{eqnarray*}
status\_ambulance &=& \{alpha1 \rightarrow FREE, alpha2 \rightarrow FREE, alpha3 \rightarrow FREE, \\ 
	&& mike1 \rightarrow FREE, mike2 \rightarrow FREE\} \\
choice\_mobilization &=& \{\} \\
mob\_mobilization &=& \{\} \\
\end{eqnarray*}
La sortie attendue est 
\begin{itemize}
	\item $incidents! = \{incident1, incident2, incident3, incident4\}$
\end{itemize}

\subsubsection{Troisième test}
Tous les incidents sont affectés
\begin{eqnarray*}
registered\_ambulance &=& \{alpha1, alpha2, alpha3, mike1\}\\
position\_ambulance &=& \{alpha1 \rightarrow (0,0), alpha2 \rightarrow (0,0), alpha3 \rightarrow (0,0), \\ 
	&& mike1 \rightarrow (0,0)\}\\
kind\_ambulance &=& \{alpha1 \rightarrow NORMAL, alpha2 \rightarrow NORMAL, alpha3 \rightarrow NORMAL, \\ 
	&& mike1 \rightarrow MEDICALIZED\} \\
\\
status\_ambulance &=& \{alpha1 \rightarrow MOBILIZED, alpha2 \rightarrow MOBILIZED, alpha3 \rightarrow CHOSEN, \\ 
	&& mike1 \rightarrow MOBILIZED\} \\
choice\_mobilization &=& \{alpha3 \rightarrow incident1\} \\
mob\_mobilization &=& \{alpha1 \rightarrow incident2, alpha1 \rightarrow incident4, mike1 \rightarrow incident3\} \\
\end{eqnarray*}
La sortie attendue est 
\begin{itemize}
	\item $incidents! = \{\}$
\end{itemize}

\subsubsection{Quatrième test}
Aucun incident n'existe dans le système
\begin{eqnarray*}
registered\_incident &=& \{\}\\
victimPregnant\_incident &=& \{\}\\
localisation\_incident &=& \{\}\\
description\_incident &=& \{\}\\
position\_incident &=& \{\}\\
ambulanceKindNeeded\_incident &=& \{\}\\
\\
status\_ambulance &=& \{alpha1 \rightarrow FREE, alpha2 \rightarrow FREE, alpha3 \rightarrow FREE, \\ 
	&& mike1 \rightarrow FREE, mike2 \rightarrow FREE\} \\
choice\_ambulance &=& \{\} \\
mob\_mobilization &=& \{\} \\
\end{eqnarray*}
La sortie attendue est 
\begin{itemize}
	\item $incidents! = \{\}$
\end{itemize}

\subsubsection{Cinquième test}
Plus d'un incident dans le système n'est pas affecté
\begin{eqnarray*}
status\_ambulance &=& \{alpha1 \rightarrow FREE, alpha2 \rightarrow CHOSEN, alpha3 \rightarrow FREE, \\ 
	&& mike1 \rightarrow FREE, mike2 \rightarrow FREE\} \\
choice\_ambulance &=& \{alpha2 \rightarrow incident1\} \\
mob\_mobilization &=& \{alpha3 \rightarrow incident4\} \\
\end{eqnarray*}
La sortie attendue est 
\begin{itemize}
	\item $incidents! = \{incident2, incident3\}$
\end{itemize}

\chapter*{Conclusion}

Grâce à notre travail de formalisation effectué dans l'analyse des besoins, 
la formalisation en langage Z a été facilement déduite et construite. Les tests
ont été dérivés sans difficulté des opérations. 

Nous n'avons pas identifié d'erreur dans notre analyse des besoins par
rapport à cette formalisation, probablement du au fait qu'une grande partie
de la formalisation avait été faite préalablement, nous permettant d'identifier
ces erreurs en amont.


\end{document}
