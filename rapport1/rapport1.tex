\documentclass{report}

\usepackage[utf8]{inputenc}
\usepackage{style-report}

% L'en-tête et la mise en page sera différente
% Merci de ne pas faire de mise en page dans le document !
\title{LAS : London Ambulance System}
\subtitle{Analyse des besoins}
\author{\normalsize{Groupe 2}\\
\footnotesize{
Simon Busard, \\
Antoine Cailliau, \\
Laurent Champon,\\
Erick Lavoie, \\
Quentin Pirmez,\\
Frederic Van der Essen, \\
Géraud Talla Fotsing}}

\date{\today}

\newlength{\realtextwidth}
\setlength{\realtextwidth}{\textwidth}
\newcommand{\insertfiguremargin}[2]{
	\begin{figure}[!h]
	\noindent\begin{minipage}[!h]{\marginparsep+\marginparwidth+\marginparpush+\textwidth}
		\begin{minipage}[t]{\realtextwidth}
			\vspace{0pt}
			\includegraphics[width=\realtextwidth]{images/#1.png}
		\end{minipage}
		\hspace{\marginparsep}
		\begin{minipage}[t]{\marginparwidth+\marginparpush}
			\vspace{0pt}
			\setcaptionwidth{\marginparwidth+\marginparpush}
			\caption{#2}\label{fig:#1}
			\setcaptionwidth{0.9\realtextwidth}
		\end{minipage}		
	\end{minipage}
	\end{figure}

}

\setlength{\realtextwidth}{\textwidth}
\newcommand{\insertfigure}[2]{
	\begin{figure}[!h]
	\noindent\begin{minipage}[!h]{\marginparsep+\marginparwidth+\marginparpush+\textwidth}
		\begin{minipage}[t]{\textwidth}
			\setcaptionwidth{\textwidth}
			\caption{#2}\label{fig:#1}
			\setcaptionwidth{0.9\realtextwidth}
		\end{minipage}
		
		\vspace{0.5cm}
		\begin{minipage}[t]{\textwidth}
			\vspace{0pt}
			\includegraphics[width=\textwidth]{images/#1.png}
		\end{minipage}
	\end{minipage}
	\end{figure}
}


\setlength{\realtextwidth}{\textwidth}
\newcommand{\insertfiguredown}[2]{
	\begin{figure}[!b]
	\noindent\begin{minipage}[t]{\marginparsep+\marginparwidth+\marginparpush+\textwidth}
		\begin{minipage}[t]{\textwidth}
			\vspace{0pt}
			\includegraphics[width=\textwidth]{images/#1.png}
		\end{minipage}
		
		\vspace{0.5cm}
		\begin{minipage}[t]{\textwidth}
			\setcaptionwidth{\textwidth}
			\caption{#2}\label{fig:#1}
			\setcaptionwidth{0.9\realtextwidth}
		\end{minipage}
	\end{minipage}
	\end{figure}
}

\definecolor{rememberbackground}{rgb}{0.96,0.97,0.98}
\newcommand{\remember}[1]{
\colorbox{rememberbackground}{%
\noindent%
\makebox[\textwidth+\marginparsep+\marginparwidth][l]{
\parbox{\textwidth+\marginparsep+\marginparwidth}{
#1
}}}}

\widowpenalty=10000
\clubpenalty=10000


\begin{document}

\renewcommand{\figurename}{\sf Figure}
\setlength{\parskip}{1em}


\maketitle
\setcounter{tocdepth}{1}
\tableofcontents

\chapter*{Introduction}
	La méthode du \textit{Requirement Engineering} permet, entre autre, 
d'aborder des  problématiques complexes afin d'en tirer l'essentiel et l'utile pour la 
conception de logiciels critiques. Dans le cadre du cours de génie logiciel, il
nous est demandé de concevoir un logiciel de gestion d'ambulance similaire
au logiciel utilisé à Londres il y a quelques années.

Ce rapport présente la première partie du projet: l'analyse des besoins. Nous
avons utilisé une modélisation par buts pour guider l'ensemble de l'analyse.
Le rapport est découpé en deux parties : la première explique la manière
dont nous avons construit notre modèle et illustre les liens inter-modèles
existants. La seconde partie liste les buts et les objets modélisé à l'aide
du logiciel Objectiver.

	
\chapter{Analyse}
	Ce chapitre présente l'analyse qui a été faite du domaine. Le tout est 
volotairement découpé en multiple section afin de présenter le plus 
clairement possible les liens existants entre les modèles. 

La section \ref{section:buts} présente le modèle de buts, base du reste de 
la construction.
La section \ref{section:objets} présente les objets manipulés et utilisés 
dans les modèles. La section \ref{section:comportement} présente
les scénarios ainsi que les machines à états définies dans le cadre de
cette analyse. 
La section \ref{section:operations} présente les différentes opérations 
afférants aux buts et exécutées par les agents.
La section \ref{section:obstacles} présente enfin les obstacles, point
de départ pour une seconde itération.
	
\section{Modèle de buts}\label{section:buts}
	Dans cette partie, nous vous présentons le modèle de buts. En 
particulier, nous vous présentons la formalisation des états impliqués
par les buts que nous avons
utilisés pour découler de manière rapide et cohérente les opérations, les
machines à états et augmenter la cohérence de notre modèle de buts.

\subsection{Première couche}

	Cette section présente les divers buts de plus haut niveau. Ces buts
	sont les buts principaux de notre système. 

	\insertfiguredown{goal1}{Les buts de plus haut niveau considéré par notre
	système. Les buts présentés en jaune sont les buts qui sont affectés
	à des agents humains, les buts bleus sont des buts potentiellement
	affecté à des agents logiciel. Ici, nous pouvons voir que le témoin
	est en charge de faire l'appel au centre de secours}

	% TODO: remove victim from callreceived to be coherent with
	% the rest of the models

	Le but principal du système dans son ensemble général est
	d'assurer que toutes les personnes blessées recoivent des soins 
	appropriés. Bien entendu, cela n'est pas possible pour notre
	système car ce dernier ne peut être au courant des moindres blessures
	et autres nécessaire par l'ensemble des gens vivant sur la planète.
	Nous sommes donc contraints de considérer un sous-ensemble de ces personnes,
	de par le manque d'observabilité. 
	
	Nous considérons donc l'ensemble des personnes dont un témoin procède à 
	un appel à un centre de secours. Par simplification de notre modèle, nous
	considérons que l'appel est toujours passé par le témoin et que ce dernier
	n'est pas la victime. Il est évident que cela peut facilement être étendu
	à des cas plus générique où la vicitime est elle-même le témoin de son
	accident. Nous avons également considéré que les accidents n'impliquaient
	qu'une seule personne: notre système ne gérera donc pas les accidents où
	le nombre de vicitime est important. Enfin, nous avons également choisis
	de ne considérer qu'un modèle idéal où les appels dupliqués n'existe pas.

	\noindent\remember{
	\singlespacing
	\begin{equation*}
		\begin{array}{rcl}
			AppropriateMedicalCareGiven(i:Incident)
			& = & \exists AmbulanceMobilized(a, i), \\
			&   & \wedge AmbulanceOnScene(a, i) \\
			&   & \wedge \#a.medicalCare = 1 \\
			&   & \wedge \#i.victim.medicalCare = 1 \\
			&   & \wedge a.medicalCare \\
			&   & \rightarrow i.victim.medicalCare \\ 
			CallReceived(i:Incident)
			& = & \exists c: Call, \\
			&   & c.about \rightarrow i \\
			&   & \wedge \#c.about = 1 \\
			&   & \wedge \#i.about = 1 \\
		\end{array}
	\end{equation*}
	\onehalfspacing
	}
	
\newpage
\subsection{Seconde couche}
	
	La \og seconde couche \fg\ est constituée des enfants des buts
	feuilles de la première couche. Les buts qui sont considérés dans
	cette section font partie du raffinement du buts de haut niveau
	qui peut être remplis par notre système. Une représentation 
	graphique de ce sous-modèle est présenté à la figure \ref{fig:goal2}.
	
	Le rafinement utilisé est un rafinement par jalon. Les étapes que nous
	avons identifées sont les suivantes :
	
	\noindent\remember{
	\begin{equation*}\begin{array}{ll}
		\text{Call received} & \longrightarrow 
		\text{Incident informations known} \longrightarrow \cdots \\ 
		& \cdots  \longrightarrow \text{Ambulance mobilized} \longrightarrow 
		\text{Appropriate media care given}
	\end{array}\end{equation*}
	}
	
	L'intitulé du but \textit{Achieve[AmbulanceChosen]} n'est pas 
	exactement complet. Pour être complet, le but devrait s'appeller
	\textit{Achieve[\-Ambulance\-Chosen\-When\-Availability\-Known\-And\-Ambulance\-Kind\-Known\-And\-Accurate\-Ambulance\-Position\-Known]}. Pour des raisons évidentes 
	de concision et de lisibilité, l'intitulé court a été conservé.

	\insertfiguredown{goal2}{On peut remarquer sur cette figure les différents
	noeuds du jalon illustré ci-dessus. Il est également intéressant
	de remarquer que dans les raffinements, certains but sont déjà assignés à 
	des agents humains ou logiciel.}
	
	Le premier but feuille est le but responsable du transfert d'information
	entre le témoin, logiquement sur place, et le centre de secours. La
	bonne réalisation de ce but dépends à la fois des agents humains (témoin,
	standardiste, etc.) et de la partie logicielle responsable de 
	la gestion des informations relatives aux incidents au sein du système.
	Les différents états relatifs à ces buts sont précisés de manière formelle
	dans l'encadré suivant.
	
	\noindent\remember{
	\singlespacing
	\begin{equation*}
		\begin{array}{rcl}
			IncidentInfoKnown(i:IncidentInfo)  
			& = & IncidentInfoProcessed(i) \\
			
			AmbulanceMobilized(a: Ambulance) 
			& = & \exists i:Incident, \\
			&   & MobilizationOrderConfirmed(a, i) \\
		\end{array}
	\end{equation*}
	\onehalfspacing
	}
		
	La formalisation des états atteints par les buts issus du raffinement
	du but \textit{Achieve[IncidentInfoKnownWhenCallReceived]} sont présenté
	dans l'encadré suivant.
	
	\noindent\remember{
	\singlespacing
	\begin{equation*}
		\begin{array}{rcl}
			IncidentInfoRecorded(c:Call)  
			& = & \exists j:IncidentInfo, \exists i:Incident, \\
			&   & c.about \rightarrow i \\
			&   & \wedge j.reporting \rightarrow c \\
			&   & \wedge j.localisation \neq nul \\
			&   & \wedge j.description \neq null \\
			&   & \wedge j.victimAge \neq null \\
			&   & \wedge j.victimPregrant \neq null \\
			&   & \wedge j.id \neq null \\
			
			IncidentInfoProcessed(j:IncidentInfo)  
			& = & j.pos \neq null \\
			&   & \wedge j.ambulanceKindNeeded \neq null \\
			&   & \exists c:Call, \exists i:Incident, \\
			&   & (c.about \rightarrow i \wedge c.reporting \rightarrow j) \\
		\end{array}
	\end{equation*}
	\onehalfspacing
	}
	
	La formalisation des états atteints par les buts issus du raffinement
	de \textit{Achieve[AmbulanceMobilizedWhenIncidentInfoKnown]}
	sont présenté, comme précédemnt, dans l'encadré qui suit.
	
	\noindent\remember{
	\singlespacing
	\begin{equation*}
		\begin{array}{rcl}
			AmbulanceChosen(i:IncidentInfo)  
			& = & \exists a:AmbulanceInfo \\
			&   & \wedge a.choice = 1 \\
			&   & \wedge \#i.choice = 1 \\
			&   & \wedge a.choice \rightarrow i \\
			
			\multicolumn{3}{l}{MobilizationOrderTransmitted(a: Ambulance, i:Incident)} \\ 
			& = & \exists m: MobilizationOrder, \\
			&   & m.sender \rightarrow a \\
			&   & \wedge  m.incidentId = i.id \\
			
			\multicolumn{3}{l}{MobilizationOrderConfirmed(a:Ambulance, i:Incident)} \\ 
			& = & \exists m:MoblisationOrder: \\
			&   & m.sender \rightarrow a \wedge m.incidentId = i.id \\ 
			&   & \wedge m.acknowledgement = True \\
		\end{array}
	\end{equation*}
	\onehalfspacing
	}
	
	Enfin, un seul état relatifs aux buts n'a pas encore été formellement
	définit. Cet état est issus des buts du raffinement du but \\
	\textit{Achieve[AppropriateMedicalCareGivenWhenAmbulanceMobilized]}
	
	\noindent\remember{
	\singlespacing
	\begin{equation*}
		\begin{array}{rcl}
			AmbulanceOnScene(a:Ambulance, I:Incident)
			& = & a.pos \approxeq i.pos \\ 
		\end{array}
	\end{equation*}
	\onehalfspacing
	}


\subsection{Troisième couche}

	La troisième couche est constituée des buts enfants
	des buts de la seconde couche. Les figures \ref{fig:goal3}, \ref{fig:goal4},
	et \ref{fig:goal5} reprennent les sous-arbres des buts. Aussi, nous
	présentons, comme pour les précédente section,
	la formalisation de certains des états atteints ou relatifs aux buts.

	\insertfigure{goal3}{Nous pouvons voir sur cette
	figurer le raffinement du but \textit{AmbulanceChosenWhenIncidentInfoKnown}
	ainsi que les différentes affectations de ces buts aux buts feuilles.}
	
	\noindent\remember{
	\singlespacing
	\begin{equation*}
		\begin{array}{rcl}
		AmbulanceAvailabilityKnown(a: Ambulance)
		& = & \exists b: AmbulanceInfo, \\
		&   & b.id = a.id \\
		&   & \wedge \#b.mobilized = \#a.mobilized \\

		AmbulanceKindKnown(a: Ambulance)
		& = & \exists b: AmbulanceInfo, \\
		&   & b.id = a.id \\
		&   & \wedge typeof(a) = b.kind \\

		AmbulancePositionKnown(a:Ambulance)
		& = & \exists b:AmbulanceInfo, \\
		&   & b.id = a.id \\
		&   & \wedge a.pos \approxeq b.pos \\
		
		PositionAccurate(a:Ambulance)
		& = & \exists b:AmbulanceInfo, \\
		&   & a.id = b.id \\
		&   & \wedge b.pos = a.pos \\
		\end{array}
	\end{equation*}
	\onehalfspacing
	}

	\insertfigure{goal4}{Nous pouvons voir sur cette figurer
	le raffinement du but reponsable de l'envoi de l'ordre de mobilisation. 
	On peut voir que les agents ne sont pas situés dans le même monde: 
	l'InformationProcessor est situé du coté de la central d'appel tandis
	que le MDT (Mobile Data Terminal) est situé dans l'ambulance.}
	\insertfigure{goal5}{Les buts illustrés sont issus du raffinement du but
	responsable de la confirmation de la mobilisation. On y voit les
	buts qui sont à la charge de l'équipage de l'ambulance et du terminal
	informatique placé dans l'ambulance.}


	
\section{Modèle des objets}\label{section:objets}
	\subsection{Diagramme des objets}

	Dans cette partie, nous présentons la conception de notre modèle objet. Ce
	modèle est un modèle du domaine c'est-à-dire que les classes, associations
	et attributs correspondent à des éléments du domaine (éventuellement
	partagés avec le logiciel), apparus lors de l'analyse des objectifs. 

	Au cours de l'élaboration du diagramme d'objets, nous avons dû déterminer
	quelles classes, quelles associations et quels attributs du domaine devaient avoir une
	image manipulée par le logiciel comme données persistantes dans une base de
	données conceptuelle, et comme classe abstraite intervenant dans des
	composants de l'architecture à développer. 

	À partir de notre modèle de buts nous avons donc abouti à un modèle d'objets
	pour le moins solide et respectant ainsi la cohérence intermodèle. 

	Les classes représentant les éléments du domaine sont illustrées dans la
	figure \ref{fig:Objects2}. Certaines d'entre elles ont une "image" dans le système, c'est
	le cas de la classe Incident et Ambulance dont les classes "miroirs"
	respectives sont IncidentInfo et AmbulanceInfo (figure  \ref{fig:Objects1}). Il existe donc
	une association "tracking" entre ces objets réels et leur représentation
	dans le système. Cette association existe grâce aux objets que l'on nomme
	"interface" (figure  \ref{fig:Objects3}) et qui assurent la communication entre les objets du
	monde réel et leur représentation dans le système. 

	Cette démarche nous pousse à introduire de nouveaux buts non fonctionnels
	notamment de précision, exprimant que l'état de la base de données du
	logiciel doit refléter fidèlement l'état des objets/associations de
	l'environnement que les objets/associations logiciels représentent. 

	\insertfigure{Objects2}{Objets appartenant à l'environnement}
	\insertfigure{Objects3}{Objets à l'interface de l'environnement et du logiciel}
	\insertfigure{Objects1}{Objets appartenant au logiciel}
	\insertfigure{ObjectsGlobal}{Schéma de l'ensemble des objets}

\subsection{Spécification des concepts}


\section{Modèle des agents}\label{section:agents}
	\subsection{Diagramme de contexte}

	Le diagramme de contexte met en lumière les différentes interactions entre 
	agents du système au travers des variables qu'ils contrôlent et monitorent. 
	Il est construit à partir du modèle de buts (où chaque but feuille est 
	assigné à un agent) et du modèle objet.
	
		% TODO : supprimer ce diagramme?  Le mettre en annexe?
		% Diagramme probablement à revoir au lu des commentaires de l'assistant
		% (voir mail)
		\insertfigure{context}{Diagramme de contexte complet}
	
	L'intérêt principal du diagramme de contexte est de représenter les interactions
	à l'intérieur du logiciel (interactions agent logiciel/agent logiciel) et
	les interactions à l'interface (interactions agent logiciel/agent
	d'environnement).
	
	Dans le cas où des noms de classes sont utilisées au lieu d'attributs ou d'association, il est sous-entendu que l'agent qui contrôle les attributs de la classes mentionnée les contrôle tous et l'agent qui monitore les monitore tous.  Cette notation a été utilisée pour rendre le diagramme plus concis.
	
	\insertfigure{context_soft-interface}{Diagramme de contexte des interactions
	 logicielles et d'interface}

	
\section{Modèle de comportement}\label{section:comportement}
	Cette section présente les scénarios (\ref{sec:scenario}) ainsi que
les machines à états (\ref{sec:MachineAEtat}) des ambulances et des
incidents.

\subsection{Scénario}
\label{sec:scenario}

	Les scénarios les plus intéressants pour la compréhension du modèle sont
	fournis dans cette section.  Dans les cas où les buts feuilles semblaient
	trop simples pour obtenir un scénario digne d'intérêt, les buts parents ont
	été utilisés.  Notons ici que nous avons omis le scénario correspondant au
	but Maintain[AmbulanceKindKnown] car le type d'ambulance est assumé
	constant.  Cette situation ne génère alors pas de communication entre les
	différents agents. 

	Pour chacun des scénarios présentés, le but correspondant est mentionné.
	De plus, les états sous-entendus pour chacun des scénarios sont présentés
	en marge des différents scénarios, en turquoise pour les états de
	l'incident et en gris/violet pour les états de l'ambulance.  Ces états
	correspondent aux états présentés dans la section \ref{sec:MachineAEtat}.

	%\insertfiguremargin{Achieve[CallReceivedWhenInjuredPeople]}{Le scénario illustre le but \textit{Achieve[Call\-Received\-When\-Injured\-People]}, on peut y voir les interactions
	%entre les différents agents humains, au début de la chaîne.}
	
	\insertfiguremargin{Achieve[IncidentInfoKnownWhenCallReceived]}{Ce scénario 
	illustre le but \textit{Achieve[Incident\-Info\-Known\-When\-Call\-Received]}. 
	On peut y voir le début de la chaîne.}
	
	\insertfiguremargin{Maintain[AmbulanceAvailabilityKnown]}{Ce scénario illustre la
	mobilisation d'une ambulance jusqu'au soins prodigués à la victime. Les bulles
	bleues indiquent les états de l'\textit{ambulance} et de l'\textit{ambulanceInfo}, les bulles
	mauves quant à elles illustrent les états de l'\textit{incident} et de l'\textit{incidentInfo}}
	
	%\insertfiguremargin{Maintain[AmbulancePositionKnownAndAccurate]}{Maintain[AmbulancePositionKnownAndAccurate]}
	
	\insertfiguremargin{Achieve[AmbulanceMobilizedWhenIncidentInfoKnown]}{Ce scénario illustre la mobilisation d'une ambulance.}
	
	%\insertfiguremargin{Achieve[AppropriateMedicalCareGivenWhenAmbulanceMobilized]}{Achieve[AppropriateMedicalCareGivenWhenAmbulanceMobilized]}	

\subsection{Machines à état}
\label{sec:MachineAEtat}

	La section présente une généralisation des scénarios présentés dans la
	section \ref{sec:scenario}.   Tel que suggéré par le tuteur,  les machines
	à état dans le cas de l'ambulance et de l'incident correspondent à
	l'aggrégation des états des objets réels et des \textit{tracking objects}.
	De plus, il n'y a pas de distinction réalisée entre \textit{AmbulanceCrew}
	et \textit{Ambulance}.  Ces choix ont été réalisés pour mettre l'emphase
	sur le comportement du système et éviter la complexité liée à la présence
	des objets nécessaires à la transmission de l'information entre
	l'environnement et le logiciel.
 	
	\insertfiguremargin{statechart_ambulance}{États de l'Ambulance et de l'AmbulanceInfo}

	\insertfiguremargin{statechart_incident}{États de l'Incident et l'IncidentInfo}


\section{Modèle des opérations}\label{section:operations}
	Cette section présente un ensemble d'opération. Nous avons choisi les quatres
opérations correspondant à des buts feuilles et assignées à des agents logiciels
à développer.

Afin d'illustrer les liens entre les modèles, nous reprenons le nom du but
qui est opérationnalisé par l'opération, le nom de l'agent qui effectuera 
l'opération, l'état à atteindre après cette opération et enfin, l'évènement
attaché à cette opération.

\subsection{processIncidentInfo}
	
	\insertfigure{op1}{}
	
	\begin{table}[!h] \centering
		\begin{tabularx}{\textwidth}{|l|X|} \hline
			Goal & Achieve[IncidentInfoProcessedWhenIncidentInfoRecorded] \\ \hline
			Agent & InfoProcessor \\ \hline
			Goal state & IncidentInfoProcessed \\ \hline
			Event & IncidentInfoProcessing \\ \hline
			In & $i: IncidentInfo$ \\ \hline
			Out & $i: IncidentInfo$ \\ \hline
			Pre & $\exists c: Call, \exists j: Incident (c.about \rightarrow j \wedge c.reporting \rightarrow i)$ \\ \hline
			Post & $i.pos != '' \wedge i.ambulanceKindNeeded != ''$  \\
				 & La position (pos) contenue dans i correspond à la position (localisation) de i sous forme exploitabular par le système 
				  Le type d'ambulance nécessaire de i est calculé selon les règles données par le gouvernement et sur base des 
				  informations présentent dans i \\ \hline
		\end{tabularx}
		\caption{processIncidentInfo}
	\end{table}

\subsection{recordAccurateAmbulancePosition}
	
	\insertfigure{op2}{}
	
	\begin{table}[!h] \centering
		\begin{tabularx}{\textwidth}{|l|X|} \hline
			Goal & Achieve[AccurateAmbulancePositionRecorded\\ &WhenAccurateAmbulancePositionSent] \\ \hline
			Agent & AmbulanceTracker \\ \hline
			Goal state & AmbulancePositionAccurate and AmbulancePositionKnown \\ \hline
			Event & AccurateAmbulancePositionRecording \\ \hline
			In & $a: Ambulance$ \\ \hline
			Out & $b: AmbulanceInfo$ \\ \hline
			Pre & $\exists b: ambulanceInfo : a.id = b.id$ \\ \hline
			Post & $b.pos = a.pos$ \\ \hline
		\end{tabularx}
		\caption{recordAccurateAmbulancePosition}
	\end{table}

\subsection{choseAmbulance}
	
	\insertfigure{op3}{}
	
	\begin{table}[!h] \centering
		\begin{tabularx}{\textwidth}{|l|X|} \hline
			Goal & Achieve[AmbulanceChosenWhen\\ & AvailabilityKnownAnd \\ & AmbulanceKindKnownAnd \\ & AccurateAmbulancePositionKnown] \\ \hline
			Agent & InfoProcessor \\ \hline
			Goal state & AmbulanceChosen \\ \hline
			Event & AmbulanceChoice \\ \hline
			In & $i: IncidentInfo$ \\ \hline
			Out & $a: AmbulanceInfo$ \\ \hline
			Pre & $\exists a: AmbulanceInfo : \#a.mobilisation = 0 \wedge \#a.choice = 0$ \\ \hline
			Post & $\#a.choice = 1 \wedge \#i.choice = 1 \wedge i.choice \rightarrow a$ \\ \hline
		\end{tabularx}
		\caption{choseAmbulance}
	\end{table}

\subsection{sendMobilizationOrder}
	
	\insertfigure{op4}{}
	
	\begin{table}[!h] \centering
		\begin{tabularx}{\textwidth}{|l|X|} \hline
			Goal & Achieve[MobilizationOrderSentWhenBestAmbulanceChosen] \\ \hline
			Agent & InfoProcessor \\ \hline
			Goal state & MobilizationOrderTransmitted \\ \hline
			Event & MobilizationOrderTransmittion \\ \hline
			In & $a: Ambulance, i: Incident$ \\ \hline
			Out & $m: MobilizationOrder$ \\ \hline
			Pre & $\#a.choice = 1 \wedge a.choice \rightarrow i$ \\ \hline
			Post & $\exists m: MobilizationOrder : m.ambulance = a \wedge m.incident = i$ \\ \hline
		\end{tabularx}
		\caption{sendMobilizationOrder}
	\end{table}


\section{Résolution d'obstacles}\label{section:obstacles}
	\section{Achieve[AccurateAmbulancePositionRecordedWhenAccurateAmbulancePositionSent]}
	\insertfigure{Obstacle3}{Diagramme de résolution d'obstacle}
	Le but qui nous concerne ici est d'assurer que la précision des
	données envoyées est garantie à la réception et que ces données
	puissent être enregistrées. Les obstacles et leur résolution coulent 
	de source.

\section{Achieve[AccurateAmbulancePositionSent]}
	\insertfigure{Obstacle2}{Diagramme de résolution d'obstacle}
	Ce but correspond à l'envoi des données de l'AVLS et de s'assurer
	que celles-ci sont correctes. 

	L'obstacle dont la résolution est la plus intéressante est le non
	fonctionnement du matériel responsable de l'exécution de ce but.

	La résolution distingue deux cas, celui où elle est libre, et celui
	où elle est mobilisée. Dans le premier cas, on envoie directement
	l'ambulance à une station où elle sera réparée. Dans le second
	cas, l'ambulance effectue son travail et retourne ensuite à la 
	station pour réparation.

	Cette résolution est incomplète car notre modèle des buts et objet
	ne permet pas de considérer les cas suivant : 
	\begin{itemize}
		\item Empècher de choisir ou de mobiliser l'ambulance lorsqu'elle
		doit être réparée ou en réparation,
		\item Ne pas mobiliser une ambulance choisie si elle a eu un
		problème entre temps.
	\end{itemize}

	De plus, d'autres problèmes apparaissent; L'état "AmbulanceOnScene"
	ne saura être détecté par le systême informatique dans le
	cas d'une ambulance disfonctionelle.

	Pour résoudre ces problèmes il faut ajouter un attribut broken
	à l'ambulance et modifier les définitions des buts de choix et
	de mobilisation de l'ambulance. 

	De plus l'AVLS et le systême de transmissions ne sont sans doute pas
	les seuls appareils de l'ambulance pouvant tomber en panne, le but
	de réparation correspond très certainement à un but de général de
	plus haut niveau consistant à maintenir l'ambulance en état de marche.

\section{Achieve[AmbulanceOnSceneWhenAmbulanceMobilized]}
	\insertfigure{Obstacle1}{Diagramme de résolution d'obstacle}
	Ce but correspond à l'arrivée de l'ambulance sur les lieux indiqués
	de l'incident après sa mobilisation.
	
	Le raffinement consistant à démobiliser une ambulance accidentée ou
	coincée peut sembler superflu, mais est justifié par le respect des
	cardinalités de notre modèle objet. 

	Le but \emph{Achieve[MedicalCareGivenByOtherAmbulanceWhenAmbulanceDemobilized]}
	n'est manifestement pas complet ni suffisemment raffiné.
	
	En fait, deux alternatives s'offrent à nous. La première et la moins bonne
	consiste à répliquer l'entièreté de l'arbre des buts \emph{Achieve[MedicalCareGivenWhenAmbulanceMobilized}
	comme raffinement du précédent. L'autre alternative consiste à modifier
	les buts parents afin qu'ils se chargent de mobiliser l'ambulance de 
	secours. Pour cela, il faut transformer \emph{Achieve[AmbulanceMobilizedWhenIncidentInformationKnown]}
	par un \emph{Maintain[WorkingAnbulanceMobilizedWhenIncidentInformationKnownAndNotResolved]}
	Ce but s'assurera donc qu'il y a toujours une ambulance en état de marche 
	mobilisée pour chaque incident. 
	
	Cependant cela nécéssite de définir deux états supplémentaires:  
	Working(Ambulance) et Resolved(IncidentInformation). L'ajout de ces deux
	états nécessite de modifier notre modèle objet et d'ajouter de nouveaux 
	buts et raffinements, à identifier lors d'une deuxième passe d'analyse
	et de réflexion sur notre modêle.

	Enfin, Cette résolution n'est pas non plus complète. Il faut en effet
	modifier la représentation de la carte du systême de routage et de choix
	d'ambulance pour que celui ci n'envoie pas une deuxième ambulance dans
	un embouteillage. 

	Pour finir, il faut aller rechercher l'ambulance accidentée et la 
	réparer. 

	Ces deux buts sont à rattacher comme raffinements de buts plus généraux
	qui ne se trouvent pas dans notre modèle actuel. Cela encore révèle la
	nécessité d'une seconde passe d'analyse.



	
	
\chapter{Cahier des charges}
	Ce chapitre définit de manière complète les buts et les objets utilisé
	dans la modélisation.

	\section{Définition des buts}\label{section:buts2}
	\subsection{Définitions de buts}


Achieve[AccurateAmbulancePositionRecordedWhenAccurateAmbulancePositionSent]

The accurate position of the ambulance is eventually recorded in the system when ambulance position is sent.


Achieve[AccurateAmbulancePositionSent]

The accurate position of the ambulance is eventually sent.


Achieve[AmbulanceChosen]

The best ambulance is eventually chosen. The best ambulance is the available ambulance of the right kind which may arrive quickly on the scene.


Achieve[AmbulanceChosenWhenIncidentInfoKnown]

All available ambulances of the right kind are eventually chosen.


Achieve[AmbulanceMobilizedWhenIncidentInfoKnown]

Once the information about the incident are known, the right ambulance and his crew are eventually mobilized.


Achieve[AmbulanceOnSceneWhenAmbulanceMobilized]

Ambulance eventually arrived on the scene when the ambulance is mobilized by the system.


Achieve[AppropriateMedicalCareAreGivenQuickly]

According to the standard, appropriate medical care are given quickly. An ambulance must arrive at the scene within 14 of the call commencing on 95\% of occasion. An ambulance must arrive at the scene within 8 minutes of the call commencing on 50\% of occasions.


Achieve[AppropriateMedicalCareGivenWhenAmbulanceMobilized]

Once the ambulance is mobilized, ambulance crew  give eventually the appropriate medical care to the victim.


Achieve[AppropriateMedicalCareGivenWhenAmbulanceOnScene]

The appropriate medical care are eventually given to the victim by the ambulance crew once the ambulance arrived on scene.


Achieve[AppropriateMedicalCareGivenWhenCallReceived]

For every call received the appropriate medical care are given to the victim taking in account the time constraints


Achieve[AppropriateMedicalCareGivenWhenInjuredPeople]

Every injured people receive the appropriate medical care that he need


Achieve[CallReceivedWhenInjuredPeople]

For every injured people who need medical care, a call is received by the call center.


Achieve[IncidentInfoKnownWhenCallReceived]

For every call received all needed information about the incident are eventually known by the system.


Achieve[IncidentInfoProcessedWhenIncidentInfoRecorded]

Information about the incident are eventually processed (e.g. the avls coordinates are computed and incident gravity is defined) when the information about the incident are recorded in the system.


Achieve[IncidentInfoRecordedWhenCallReceived]

The information about the incident are eventually recorded in the system when the call is received by the call center.


Achieve[MobilisationOrderConfirmedWhenMobilisationOrderTransmitted]

Ambulance crew eventually confirmed mobilisation after order transmission.


Achieve[MobilizationOrderConfirmedByAmbulanceCrewWhenMobilizationOrderDisplayed]

When the mobilization order is displayed on the MDT, it is eventually confirmed by the ambulance crew.
 

Achieve[MobilizationOrderConfirmedWhenMobilizationOrderConfirmedByAmbulanceCrew]

When the ambulance crew confirmed the mobilization order, the confirmation is eventually confirmed to the system. 


Achieve[MobilizationOrderDisplayedWhenMobilizatedOrderTransmitted]

When mobilization order is transmitted, it is eventually displayed on the MDT.
 

Achieve[MobilizationOrderReadWhenMobilizationOrderDisplayed]

The mobilization order sent by the system to the ambulance crew is eventually read by the ambulance crew.


Achieve[MobilizationOrderReceivedWhenMobilizationOrderSent]

The mobilization order is eventually received when it was sent from the system.


Achieve[MobilizationOrderSentWhenBestAmbulanceChosen]

The mobilization order is eventually sent when the best ambulance has been choosen.


Achieve[MobilizationOrderTransmittedWhenAmbulanceChosen]

The mobilization order is enventually transmitted to the ambulance crew when the best ambulance is chosen


AmbulanceMobilizedWhenMobilizationOrderConfirmed

When the mobilization order is confirmed to the system, the ambulance is considered as mobilized. 


AmbulancePositionKnownWhenAmbulancePositionRecorded

The position of an ambulance is known when the ambulance position is recorded in the system.


IncidentInfoKnownWhenProcessed

The incident information are known when the incident information are processed by the system.


Maintain[AmbulanceAvailabilityKnown]

The ambulance availability is always known by the system.


Maintain[AmbulanceKindKnown]

The kind of the ambulance is always known by the system.


Maintain[AmbulancePositionKnownAndAccurate]

The ambulance position is always known by the system.


Maintain[HighSystemReliability]

The system is sufficantly reliable.
 

Maintain[HighSystemUsability]

The system is sufficantly usable.

 

Maintain[LowEnvironmentalImpact]
 
The system has a low environmental impact.


Maintain[LowSystemCosts]

System costs are low.
 

MobilisationOrderTransmittedWhenMobilisationOrderRecieved

The mobilization order is considered as transmitted when the mobilization order was received by the ambulance crew.


	\section{Définition des objets}\label{section:objets2}
		\subsection{Ambulance}
\begin{tabularx}{\textwidth}{|X|X|} \hline
	Explanation & Domain Hypothesis\\ \hline
	Vehicule transporting the crew to the incident location and possibly the victim to the hospital. 
	Communications between the crew and the rest of the system occurs in it. 
	& The ambulance is assumed to always be in working condition. \\ \hline	
\end{tabularx}

\subsection{\og AmbulanceCrew \fg}

\begin{tabularx}{\textwidth}{|X|X|} \hline
Explanation & Domain Hypothesis\\ \hline
Medical crew giving the medical care to the victim.  Use the ambulance for transportation. & The crew is assumed to be competent, always reachable and having everything needed to do the intervention. \\ \hline
\end{tabularx}


\subsection{AmbulanceInfo}

\begin{tabularx}{\textwidth}{|X|X|} \hline
Explanation & Domain Hypothesis\\ \hline
Tracking object maintaining all the information needed by the CAD about the physical ambulance.
& \\ \hline
\end{tabularx}


\subsection{AvailabilityMessage}

\begin{tabularx}{\textwidth}{|X|X|} \hline
Explanation & Domain Hypothesis\\ \hline
Never Used
&  Messages are always assumed to be delivered in order of sending, without any being lost or corrupted. Delay is assumed to be not significant. \\ \hline
\end{tabularx}


\subsection{\og AVLS \fg}

\begin{tabularx}{\textwidth}{|X|X|} \hline
Explanation & Domain Hypothesis\\ \hline
Physical object tracking the spatial position of the ambulance
&  Every ambulance has a fixed GPS inside \\ \hline
\end{tabularx}


\subsection{AVLSMessage}

\begin{tabularx}{\textwidth}{|X|X|} \hline
Explanation & Domain Hypothesis\\ \hline
Message giving the current position of the ambulance through the AVLS
& Messages are always assumed to be delivered in order of sending, without any being lost or corrupted Delay is assumed to be not significant. \\ \hline
\end{tabularx}


\subsection{Call}

\begin{tabularx}{\textwidth}{|X|X|} \hline
Explanation & Domain Hypothesis\\ \hline
Communication between the witness and the First Line Answerer through which the information about the incident is given.
& The call is assumed to always be completed and valid.  Duplicate calls for the same incident are assumed to be non-existant.  The information given through a call is assumed to be accurate and complete. \\ \hline
\end{tabularx}


\subsection{Incident}

\begin{tabularx}{\textwidth}{|X|X|} \hline
Explanation & Domain Hypothesis\\ \hline
Situation potentially dangerous for a victim, linked to a physical location.
& The incident is assumed to always threaten only one victim.  The location is assumed to always be reachable by an ambulance.  The situation is assumed to be stable from the time the incident happens to the time the ambulance crew provide the medical care. \\ \hline
\end{tabularx}


\subsection{IncidentInfo}

\begin{tabularx}{\textwidth}{|X|X|} \hline
Explanation & Domain Hypothesis\\ \hline
Tracking object maintaining information about the physical incident.
& As soon as the information is entered in the CAD concerning an incident, it is assumed to be complete and accurate.  Since the incident is assumed to be stable from the time the call is made to the time the medical care is given, the information used by the system is always valid. \\ \hline
\end{tabularx}


\subsection{\og MDT \fg}

\begin{tabularx}{\textwidth}{|X|X|} \hline
Explanation & Domain Hypothesis\\ \hline
Mobile Data Terminal, small screen with a keyboard connected through a wireless link to the Computer Aided Despatch System (CAD) installed in every ambulance.
& The MDT is assumed to always be working and connected to the CAD \\ \hline
\end{tabularx}


\subsection{MDTMessage}

\begin{tabularx}{\textwidth}{|X|X|} \hline
Explanation & Domain Hypothesis\\ \hline
"Physical" transport of the information from the MDT screen to the user.
& Messages are always assumed to be delivered in order of sending, without any being lost or corrupted. Delay is assumed to be not significant. \\ \hline
\end{tabularx}


\subsection{Medicalized}

\begin{tabularx}{\textwidth}{|X|X|} \hline
Explanation & Domain Hypothesis\\ \hline
Specialised ambulance for highly critical incidents.
& The kind of an ambulance is fixed and cannot change during the lifecycle of the software \\ \hline
\end{tabularx}


\subsection{MobilizationOrder}

\begin{tabularx}{\textwidth}{|X|X|} \hline
Explanation & Domain Hypothesis\\ \hline
Information sent to the ambulance crew concerning an incident. 
& Messages are always assumed to be delivered in order of sending, without any being lost or corrupted. Delay is assumed to be not significant. \\ \hline

\end{tabularx}

\subsection{Normal}

\begin{tabularx}{\textwidth}{|X|X|} \hline
Explanation & Domain Hypothesis\\ \hline
Regular ambulance equipped for most situations
& The kind of an ambulance is fixed and cannot change during the lifecycle of the software \\ \hline
\end{tabularx}


\subsection{\og Victim \fg}

\begin{tabularx}{\textwidth}{|X|X|} \hline
Explanation & Domain Hypothesis\\ \hline
Person being incommodated and/or injured due to an incident.  The person is unable to help himself/herself and need external help.
& The victim is always assumed to be different from the witness.  An incident always concerns only one victim.  Medical care needed involves always only one ambulance crew. \\ \hline
\end{tabularx}


\subsection{\og Witness \fg}

\begin{tabularx}{\textwidth}{|X|X|} \hline
Explanation & Domain Hypothesis\\ \hline
Person who witness the incident and call the system and talk to the FirstLineAnswerer
& There is always a witness for every incident, separate from the victim.  The witness is assumed to be able to provide complete and accurate information about the incident. \\ \hline
\end{tabularx}


\chapter*{Conclusion}
	Au travers de cette première partie de projet, nous avons pu nous rendre
compte de l'importance d'une méthode d'analyse rigoureuse et systématique.
Les systèmes critiques, tel qu'un logiciel de dispatching d'ambulance, demande
une grande analyse et une bonne connaissance du domaine. 

Cette première passe dans les modèles nous a permis d'acquérir une certaine
connaissance du domaine. Toutefois, nous nous sommes rendu compte de 
l'importance de faire un grand nombre d'itérations afin d'avoir un modèle 
complet et correspondant à la réalité. Actuellement, notre modèle est loin
d'être complet mais nous pensons avoir un bon embryon de départ.

L'analyse iter-modèle permet de rapidement se rendre compte des buts, des
états, des opérations, des rafinements manquants ou incomplets. Sans cette 
pluralité des modèles, il est probablement difficile d'aborder aussi facilement
et rapidement des problématiques aussi complexes.

Par ailleurs, nous nous sommes également rendu compte des difficultés que les
équipes d'analyse peuvent rencontrer telles que : les problèmes de communication
sur les définitions, la synchronisation entre les équipes, etc.

Nous aurions aimé avoir le temps de faire une autre itération sur notre modèle
ainsi que de formaliser l'ensemble du modèle. Le peu de formalisation que nous
avons fait nous a permis de nous rendre compte de ses avantages (et de ses
inconvénients). Nottement, la facilité de dérivation et de maintient de la
cohérence entre les modèles nous a probablement fort aidé dans la fin de la
première partie du projet.


\chapter*{Évaluation d'Objectiver}
	\emph{C'EST DE LA MEEEERDE}


\end{document}
