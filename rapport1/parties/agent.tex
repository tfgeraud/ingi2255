Le diagramme de contexte met en lumière les différentes interactions entre 
agents du système au travers des variables qu'ils contrôlent et monitorent. 
Il est construit à partir du modèle de buts (où chaque but feuille est 
assigné à un agent) et du modèle objet.

	% TODO : supprimer ce diagramme?  Le mettre en annexe?
	% Diagramme probablement à revoir au lu des commentaires de l'assistant
	% (voir mail)
	%\insertfiguremargin{context}{Diagramme de contexte complet}

L'intérêt principal du diagramme de contexte est de représenter les interactions
à l'intérieur du logiciel (interactions agent logiciel/agent logiciel) et
les interactions à l'interface (interactions agent logiciel/agent
d'environnement).

Dans le cas où des noms de classes sont utilisées au lieu d'attributs ou d'association, il est sous-entendu que l'agent qui contrôle les attributs de la classes mentionnée les contrôle tous et l'agent qui monitore les monitore tous.  Cette notation a été utilisée pour rendre le diagramme plus concis.

\insertfiguremargin{context_soft-interface}{Diagramme de contexte des interactions
 logicielles et d'interface}
