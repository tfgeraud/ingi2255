Au travers de cette première partie de projet, nous avons pu nous rendre
compte de l'importance d'une méthode d'analyse rigoureuse et systématique.
Les systèmes critiques, tel qu'un logiciel de dispatching d'ambulance, demande
une grande analyse et une bonne connaissance du domaine. 

Cette première passe dans les modèles nous a permis d'acquérir une certaine
connaissance du domaine. Toutefois, nous nous sommes rendu compte de 
l'importance de faire un grand nombre d'itérations afin d'avoir un modèle 
complet et correspondant à la réalité. Actuellement, notre modèle est loin
d'être complet mais nous pensons avoir un bon embryon de départ.

L'analyse iter-modèle permet de rapidement se rendre compte des buts, des
états, des opérations, des rafinements manquants ou incomplets. Sans cette 
pluralité des modèles, il est probablement difficile d'aborder aussi facilement
et rapidement des problématiques aussi complexes.

Par ailleurs, nous nous sommes également rendu compte des difficultés que les
équipes d'analyse peuvent rencontrer telles que : les problèmes de communication
sur les définitions, la synchronisation entre les équipes, etc.

Nous aurions aimé avoir le temps de faire une autre itération sur notre modèle
ainsi que de formaliser l'ensemble du modèle. Le peu de formalisation que nous
avons fait nous a permis de nous rendre compte de ses avantages (et de ses
inconvénients). Nottement, la facilité de dérivation et de maintient de la
cohérence entre les modèles nous a probablement fort aidé dans la fin de la
première partie du projet.
