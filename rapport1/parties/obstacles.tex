Le but principal de la découverte d'obstacles est de rendre l'analyse du 
problème plus complète en y incluant les cas limites et les exceptions.  La 
démarche est la suivante : prendre un but feuille (exigence ou hypothèse), 
le nier, raffiner autant que possible cette négation pour enfin résoudre chacun 
de ces raffinements.  Cela permet alors de définir de nouvelles exigences et 
hypothèses qui résolvent les cas limites et d'exception et ainsi obtenir un 
modèle plus complet et plus sûr.

Pour être complet, il faudrait obstruer chaque but feuille du modèle de buts 
idéal.  Dans le présent rapport, seuls quelques buts feuilles intéressants ont 
été obstrués.

\subsection{Achieve[AccurateAmbulancePositionRecorded When AccurateAmbulancePositionSent]}
	\insertfigure{Obstacle3}{Diagramme de résolution d'obstacle}
	Le but qui nous concerne ici est d'assurer que la précision des
	données envoyées est garantie à la réception et que ces données
	puissent être enregistrées. Les obstacles et leur résolution coulent 
	de source.
	
	La propriété du domaine \emph{Always[HighTransmissionFrequency]} est assurée 
	par une configuration correcte de l'AVLS.  \emph{Always[\-MemorySpaceFree]} 
	est, quant à elle, assurée par l'utilisation d'un matériel informatique 
	adéquat pour faire tourner le système logiciel. 

\subsection{Achieve[AccurateAmbulancePositionSent]}
	\insertfigure{Obstacle2}{Diagramme de résolution d'obstacle}
	Ce but correspond à l'envoi des données de l'AVLS et de s'assurer
	que celles-ci sont correctes. 

	L'obstacle dont la résolution est la plus intéressante est le non
	fonctionnement du matériel responsable de l'exécution de ce but.

	La résolution distingue deux cas, celui où l'ambulance est libre, et celui
	où elle est mobilisée. Dans le premier cas, on envoie directement
	l'ambulance à une station où elle sera réparée. Dans le second
	cas, l'ambulance effectue son travail et retourne ensuite à la 
	station pour réparation.

	Cette résolution est incomplète car notre modèle des buts et objet
	ne permet pas de gérer les objectifs suivants : 
	\begin{itemize}
		\item Empêcher de choisir ou de mobiliser l'ambulance lorsqu'elle
		doit être réparée ou est en réparation,
		\item Ne pas mobiliser une ambulance choisie si elle a eu un
		problème entre temps.
	\end{itemize}

	De plus, d'autres soucis apparaissent; L'état "AmbulanceOnScene"
	ne saura être détecté par le systême informatique dans le
	cas d'une ambulance dysfonctionelle.

	Pour résoudre ces problèmes il faut ajouter un attribut "broken"
	à l'ambulance et modifier les définitions des buts de choix et
	de mobilisation de l'ambulance. 

	De plus l'AVLS et le systême de transmission ne sont sans doute pas
	les seuls appareils de l'ambulance pouvant tomber en panne, le but
	de réparation correspond très certainement à un but de
	plus haut niveau consistant à maintenir l'ambulance en état de marche.

\subsection{Achieve[AmbulanceOnScene\-When\-AmbulanceMobilized]}
	\insertfigure{Obstacle1}{Diagramme de résolution d'obstacle}
	Ce but correspond à l'arrivée de l'ambulance sur les lieux indiqués
	de l'incident après sa mobilisation.
	
	Le raffinement consistant à démobiliser une ambulance accidentée ou
	coincée peut sembler superflu, mais est justifié par le respect des
	cardinalités de notre modèle objet. 

	Le but \emph{Achieve[Medical\-Care\-Given\-By\-Other\-Ambulance\-When\-Ambulance\-Demobilized]}
	n'est manifestement pas complet ni suffisamment raffiné.
	
	En fait, deux alternatives s'offrent à nous. La première et la moins bonne
	consiste à répliquer l'entièreté de l'arbre des buts 
	\emph{Achieve[MedicalCareGiven When AmbulanceMobilized]}
	comme raffinement du précédent. L'autre alternative consi\-ste à modifier
	les buts parents afin qu'ils se chargent de mobiliser l'ambulance de 
	secours. Pour cela, il faut transformer 
	\emph{Achieve[AmbulanceMobilized When IncidentInformationKnown]}
	par un \emph{Maintain[Working\-Anbulance\-Mobilized\-When\-Incident InformationKnownAndNotResolved]}
	Ce but s'assurera donc qu'il y a toujours une ambulance en état de marche 
	mobilisée pour chaque incident. 
	
	Cependant cela nécessite de définir deux états supplé\-mentaires:  
	\emph{Working(a:Ambulance)} et \emph{Resolved(i:IncidentInformation)}. 
	L'ajout de ces deux états nécessite de modifier notre modèle objet 
	et d'ajouter de nouveaux buts et raffinements, à identifier lors d'une 
	deuxième passe d'analyse et de réflexion sur notre modèle.

	Enfin, Cette résolution n'est pas non plus complète. Il faut en effet
	modifier la représentation de la carte du système de routage et de choix
	d'ambulance pour que celui-ci n'envoie pas une deuxième ambulance dans
	un embouteillage. 

	Pour finir, il faut aller rechercher l'ambulance accidentée et la 
	réparer. 

	Ces deux buts sont à rattacher comme raffinements de buts plus généraux
	qui ne se trouvent pas dans notre modèle actuel. Cela révèle une fois de
	plus la nécessité d'une seconde passe d'analyse.

