Dans cette partie, nous présentons la conception de notre modèle objet. Ce
modèle est un modèle du domaine c'est-à-dire que les classes, associations
et attributs correspondent à des éléments du domaine (éventuellement
partagés avec le logiciel), apparus lors de l'analyse des objectifs. 

Au cours de l'élaboration du diagramme d'objets, nous avons dû déterminer
quelles classes, quelles associations et quels attributs du domaine devaient avoir une
image manipulée par le logiciel comme données persistantes dans une base de
données conceptuelle, et comme classe abstraite intervenant dans des
composants de l'architecture à développer. 

À partir de notre modèle de buts nous avons donc abouti à un modèle d'objets
pour le moins solide et respectant ainsi la cohérence intermodèle. 

Les classes représentant les éléments du domaine sont illustrées dans la
figure \ref{fig:Objects2}. Certaines d'entre elles ont une "image" dans le système, c'est
le cas de la classe Incident et Ambulance dont les classes "miroirs"
respectives sont IncidentInfo et AmbulanceInfo (figure  \ref{fig:Objects1}). Il existe donc
une association "tracking" entre ces objets réels et leur représentation
dans le système. Cette association existe grâce aux objets que l'on nomme
"interface" (figure  \ref{fig:Objects3}) et qui assurent la communication entre les objets du
monde réel et leur représentation dans le système. 

Cette démarche nous pousse à introduire de nouveaux buts non fonctionnels
notamment de précision, exprimant que l'état de la base de données du
logiciel doit refléter fidèlement l'état des objets/associations de
l'environnement que les objets/associations logiciels représentent. 


Cette démarche nous pousse à introduire de nouveaux buts non fonctionnels
notamment de précision, exprimant que l'état de la base de données du
logiciel doit refléter fidèlement l'état des objets/associations de
l'environnement que les objets/associations logiciels représentent. 

\insertfigure{Objects2}{Objets appartenant à l'environnement}
\insertfigure{Objects3}{Objets à l'interface de l'environnement et du logiciel}
\insertfigure{Objects1}{Objets appartenant au logiciel}
\insertfigure{ObjectsGlobal}{Schéma de l'ensemble des objets}

\subsection{Spécification des concepts}

\begin{table}[!h] \centering
		\begin{tabularx}{\textwidth}{|l|X|X|} \hline
			Class & Explanation & Domain Hypothesis\\ \hline
			
Ambulance
 & Vehicule transporting the crew to the incident location and possibly the victim to the 				hospital.  Communications between the crew and the rest of the system occurs in it. 
& The ambulance is assumed to always be in working condition. \\ \hline	

 \og AmbulanceCrew \fg &
Medical crew giving the medical care to the victim.  Use the ambulance for transportation.
 & The crew is assumed to be competent, always reachable and having everything needed to do the intervention. \\ \hline

AmbulanceInfo
& tracking object maintaining all the information needed by the CAD about the physical ambulance.
& \\ \hline

AvailabilityMessage
&  Never Used
&  Messages are always assumed to be delivered in order of sending, without any being lost or corrupted. Delay is assumed to be not significant. \\ \hline

\og AVLS \fg
& Physical object tracking the spatial position of the ambulance
&  Every ambulance has a fixed GPS inside \\ \hline

AVLSMessage
& Message giving the current position of the ambulance through the AVLS
& Messages are always assumed to be delivered in order of sending, without any being lost or corrupted Delay is assumed to be not significant. \\ \hline

Call
& Communication between the witness and the First Line Answerer through which the information about the incident is given.
& The call is assumed to always be completed and valid.  Duplicate calls for the same incident are assumed to be non-existant.  The information given through a call is assumed to be accurate and complete. \\ \hline

Incident
& Situation potentially dangerous for a victim, linked to a physical location.
& The incident is assumed to always threaten only one victim.  The location is assumed to always be reachable by an ambulance.  The situation is assumed to be stable from the time the incident happens to the time the ambulance crew provide the medical care. \\ \hline

	\end{tabularx}
	\caption{Spécification des concepts partie 1}\label{tab:objectspec1}
\end{table}

\begin{table}[!h] \centering
		\begin{tabularx}{\textwidth}{|l|X|X|} \hline
			Class & Explanation & Domain Hypothesis\\ \hline


IncidentInfo
& Tracking object maintaining information about the physical incident.
& As soon as the information is entered in the CAD concerning an incident, it is assumed to be complete and accurate.  Since the incident is assumed to be stable from the time the call is made to the time the medical care is given, the information used by the system is always valid. \\ \hline

\og MDT \fg
& Mobile Data Terminal, small screen with a keyboard connected through a wireless link to the Computer Aided Despatch System (CAD) installed in every ambulance.
& The MDT is assumed to always be working and connected to the CAD \\ \hline

MDTMessage
& "Physical" transport of the information from the MDT screen to the user.
& Messages are always assumed to be delivered in order of sending, without any being lost or corrupted. Delay is assumed to be not significant. \\ \hline

Medicalized
& Specialised ambulance for highly critical incidents.
& \\ \hline

MobilizationOrder
& Information sent to the ambulance crew concerning an incident. 
& Messages are always assumed to be delivered in order of sending, without any being lost or corrupted. Delay is assumed to be not significant. \\ \hline

Normal
& Regular ambulance equipped for most situations
& \\ \hline

\og Victim \fg
& Person being incommodated and/or injured due to an incident.  The person is unable to help himself/herself and need external help.
& The victim is always assumed to be different from the witness.  An incident always concerns only one victim.  Medical care needed involves always only one ambulance crew. \\ \hline

\og Witness \fg
& Person who witness the incident and call the system and talk to the FirstLineAnswerer
& There is always a witness for every incident, separate from the victim.  The witness is assumed to be able to provide complete and accurate information about the incident. \\ \hline

		\end{tabularx}
	\caption{Spécification des concepts partie 2}\label{tab:objectspec2}
\end{table}
