\subsection{Diagramme des objets}

	Dans cette partie, nous présentons la conception de notre modèle objet. Ce
	modèle est un modèle du domaine c'est-à-dire que les classes, associations
	et attributs correspondent à des éléments du domaine (éventuellement
	partagés avec le logiciel), apparus lors de l'analyse des objectifs. 

	Au cours de l'élaboration du diagramme d'objets, nous avons dû déterminer
	quelles classes, quelles associations et quels attributs du domaine devaient avoir une
	image manipulée par le logiciel comme données persistantes dans une base de
	données conceptuelle, et comme classe abstraite intervenant dans des
	composants de l'architecture à développer. 

	À partir de notre modèle de buts nous avons donc abouti à un modèle d'objets
	pour le moins solide et respectant ainsi la cohérence intermodèle. 

	Les classes représentant les éléments du domaine sont illustrées dans la
	figure \ref{fig:Objects2}. Certaines d'entre elles ont une "image" dans le système, c'est
	le cas de la classe Incident et Ambulance dont les classes "miroirs"
	respectives sont IncidentInfo et AmbulanceInfo (figure  \ref{fig:Objects1}). Il existe donc
	une association "tracking" entre ces objets réels et leur représentation
	dans le système. Cette association existe grâce aux objets que l'on nomme
	"interface" (figure  \ref{fig:Objects3}) et qui assurent la communication entre les objets du
	monde réel et leur représentation dans le système. 

	Cette démarche nous pousse à introduire de nouveaux buts non fonctionnels
	notamment de précision, exprimant que l'état de la base de données du
	logiciel doit refléter fidèlement l'état des objets/associations de
	l'environnement que les objets/associations logiciels représentent. 

	\insertfigure{Objects2}{Objets appartenant à l'environnement}
	\insertfigure{Objects3}{Objets à l'interface de l'environnement et du logiciel}
	\insertfigure{Objects1}{Objets appartenant au logiciel}
	\insertfigure{ObjectsGlobal}{Schéma de l'ensemble des objets}

\subsection{Spécification des concepts}
