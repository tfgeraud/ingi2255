Dans cette partie, nous vous présentons le modèle de buts. En 
particulier, nous vous présentons la formalisation des états impliqués
par les buts que nous avons
utilisés pour découler de manière rapide et cohérente les opérations, les
machines à états et augmenter la cohérence de notre modèle de buts.

\subsection{Première couche}

	Cette section présente les divers buts de plus haut niveau. Ces buts
	sont les buts principaux de notre système. 

	\insertfigure{goal1}{Les buts de plus haut niveau}

	Le but principal du système dans son ensemble général est
	d'assurer que toutes les personnes blessées recoivent des soins 
	appropriés. Bien entendu, cela n'est pas possible pour notre
	système car ce dernier ne peut être au courant des moindres blessures
	et autres nécessaire par l'ensemble des gens vivant sur la planète.
	Nous sommes donc contraints de considérer un sous-ensemble de ces personnes,
	de par le manque d'observabilité. 
	
	Nous considérons donc l'ensemble des personnes dont un témoin procède à 
	un appel à un centre de secours. Par simplification de notre modèle, nous
	considérons que l'appel est toujours passé par le témoin et que ce dernier
	n'est pas la victime. Il est évident que cela peut facilement être étendu
	à des cas plus générique où la vicitime est elle-même le témoin de son
	accident. Nous avons également considéré que les accidents n'impliquaient
	qu'une seule personne: notre système ne gérera donc pas les accidents où
	le nombre de vicitime est important. Enfin, nous avons également choisis
	de ne considérer qu'un modèle idéal où les appels dupliqués n'existe pas.

	\singlespacing
	\begin{equation*}
		\begin{array}{rcl}
			AppropriateMedicalCareGiven(i:Incident)
			& = & \exists AmbulanceMobilized(a, i), \\
			&   & \wedge AmbulanceOnScene(a, i) \\
			&   & \wedge \#a.medicalCare = 1 \\
			&   & \wedge \#i.victim.medicalCare = 1 \\
			&   & \wedge a.medicalCare \\
			&   & \rightarrow i.victim.medicalCare \\ 
			CallReceived(i:Incident)
			& = & \exists c: Call, \\
			&   & c.about \rightarrow i \\
			&   & \wedge \#c.about = 1 \\
			&   & \wedge \#i.about = 1 \\
		\end{array}
	\end{equation*}
	\onehalfspacing

\subsection{Seconde couche}
	
	La \og seconde couche \fg est constituée des enfants des buts
	feuilles de la première couche. Les buts qui sont considérés dans
	cette section font partie du raffinement du buts de haut niveau
	qui peut être remplis par notre système.
	
	Le rafinement utilisé est un rafinement par jalon. Les étapes que nous
	avons identifées sont les suivantes :
	\begin{equation*}
	\text{Call received} \longrightarrow \text{Incident informations known} \longrightarrow \text{Ambulance mobilized} \longrightarrow \text{Appropriate media care given}
	\end{equation*}
	
	Notons que le \textit{Achieve[AmbulanceChosen]} est en fait
	\textit{Achieve[\-Ambulance\-Chosen\-When\-Availability\-Known\-And\-Ambulance\-Kind\-Known\-And\-Accurate\-Ambulance\-Position\-Known]}
	pour des raisons évidentes de concision et de lisibilité, nous avons 
	changé l'intitulé.

	\insertfigure{goal2}{Le détail des sous-buts de haut niveau.}
	
	La formalisation des états atteints par les buts feuilles du premier fils sont
	les suivants.
	\singlespacing
	\begin{equation*}
		\begin{array}{rcl}
			IncidentInfoKnown(i:IncidentInfo)  
			& = & IncidentInfoProcessed(i) \\
			
			AmbulanceMobilized(a: Ambulance) 
			& = & \exists i:Incident, \\
			&   & MobilizationOrderConfirmed(a, i) \\
		\end{array}
	\end{equation*}
	\onehalfspacing
	
	La formalisation des états atteints par les buts feuilles du second fils sont
	les suivants.
	\singlespacing
	\begin{equation*}
		\begin{array}{rcl}
			IncidentInfoRecorded(c:Call)  
			& = & \exists j:IncidentInfo, \exists i:Incident, \\
			&   & c.about \rightarrow i \\
			&   & \wedge j.reporting \rightarrow c \\
			&   & \wedge j.localisation \neq nul \\
			&   & \wedge j.description \neq null \\
			&   & \wedge j.victimAge \neq null \\
			&   & \wedge j.victimPregrant \neq null \\
			&   & \wedge j.id \neq null \\
			
			IncidentInfoProcessed(j:IncidentInfo)  
			& = & j.pos \neq null \\
			&   & \wedge j.ambulanceKindNeeded \neq null \\
			&   & \exists c:Call, \exists i:Incident, \\
			&   & (c.about \rightarrow i \wedge c.reporting \rightarrow j) \\
		\end{array}
	\end{equation*}
	\onehalfspacing
	
	La formalisation des états atteints par les buts feuilles du troisième fils sont
	les suivants.
	\singlespacing
	\begin{equation*}
		\begin{array}{rcl}
			AmbulanceChosen(i:IncidentInfo)  
			& = & \exists a:AmbulanceInfo \\
			&   & \wedge a.choice = 1 \\
			&   & \wedge \#i.choice = 1 \\
			&   & \wedge a.choice \rightarrow i \\
			
			\multicolumn{3}{l}{MobilizationOrderTransmitted(a: Ambulance, i:Incident)} \\ 
			& = & \exists m: MobilizationOrder, \\
			&   & m.sender \rightarrow a \\
			&   & \wedge m.incident = 1 \\
			
			\multicolumn{3}{l}{MobilizationOrderConfirmed(a:Ambulance, i:Incident)} \\ 
			& = & \exists m:MoblisationOrder: \\
			&   & m.sender \rightarrow a \wedge m.incidentId = i.id \\ 
			&   & \wedge m.acknowledgement = True \\
		\end{array}
	\end{equation*}
	\onehalfspacing
	
	La formalisation des états atteints par les buts feuilles du dernier fils sont
	les suivants.		
	\singlespacing
	\begin{equation*}
		\begin{array}{rcl}
			AmbulanceOnScene(a:Ambulance, I:Incident)
			& = & a.pos \approxeq i.pos \\ 
		\end{array}
	\end{equation*}
	\onehalfspacing


\subsection{Troisième couche}

	La troisième couche est constituée des buts enfants
	des buts de la seconde couche. Les figures \ref{fig:goal3}, \ref{fig:goal4},
	et \ref{fig:goal5} reprennent les sous-arbres des buts. Aussi, nous
	présentons la formalisation de certains des états atteints par les
	buts.

	\insertfigure{goal3}{Troisième niveau, raffinement du but \textit{AmbulanceChosenWhenIncidentInfoKnown}}
	\singlespacing
	\begin{equation*}
		\begin{array}{rcl}
		AmbulanceAvailabilityKnown(a: Ambulance)
		& = & \exists b: AmbulanceInfo, \\
		&   & b.id = a.id \\
		&   & \wedge \#b.mobilized = \#a.mobilized \\

		AmbulanceKindKnown(a: Ambulance)
		& = & \exists b: AmbulanceInfo, \\
		&   & b.id = a.id \\
		&   & \wedge typeof(a) = b.kind \\

		AmbulancePositionKnown(a:Ambulance)
		& = & \exists b:AmbulanceInfo, \\
		&   & b.id = a.id \\
		&   & \wedge a.pos \approxeq b.pos \\
		
		PositionAccurate(a:Ambulance)
		& = & \exists b:AmbulanceInfo, \\
		&   & a.id = b.id \\
		&   & \wedge b.pos = a.pos \\
		\end{array}
	\end{equation*}
	\onehalfspacing

	\insertfigure{goal4}{La suite du troisième niveau, raffinement du but reponsable
	de l'envoi de l'ordre de mobilisation.}
	\insertfigure{goal5}{La fin du troisième niveau, raffinement du but
	responsable de la confirmation de la mobilisation.}

