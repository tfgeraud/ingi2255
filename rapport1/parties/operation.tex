Cette section présente un ensemble d'opérations. Nous avons choisi les quatres
opérations correspondant à des buts feuilles et assignées à des agents logiciels
à développer.

Afin d'illustrer les liens entre les modèles, nous reprenons le nom du but
qui est opérationnalisé par l'opération, le nom de l'agent qui effectuera 
l'opération, l'état à atteindre après cette opération et enfin, l'évènement
attaché à cette opération.

\subsection{Traitement des données}

	L'opération \textit{processIncidentInfo} est une opération qui est effectuée par l'\textit{InformationProcessor},
	partie logicielle gérant tout le traitement des données au sein de
	l'application. Cette opération consiste à transformer les informations
	reçues par l'appellant en informations directement exploitables par le
	système informatique. Typiquement, cette opération déterminera 
	le type d'ambulance qui sera nécessaire et les positions \og informatisées \fg
	de la localisation. Par exemple, si le témoin a donné comme information
	que la victime était une femme enceinte inconsciente sur la place de l'accueil
	à louvain-la-neuve, le système calculerait qu'il faut une ambulance
	médicalisée et que l'incident à eu lieu à la position $(50.670932,4.616318)$.
	
	La figure \ref{fig:op1} illustre le but opérationnalisé par cette opération,
	en montrant où cette dernière est placée dans le diagramme de but. La figure
	\ref{fig:statechart_incident} quant à elle montre les transitions d'état 
	de l'incident et donc, la transition correspondant à l'opération. 
	L'opération est synthétisée au tableau \ref{tab:op1}.
	
	\vfill
	\insertfiguremargin{op1}{L'opération \textit{Achieve[processIncidentInfo]} opérationnalise \textit{Achive[Incident\-Info\-Processed\-When\-Incident\-Info\-Recorded]}.}
	\vfill
	
	\begin{table}
		\begin{tabularx}{\textwidth}{|l|X|} \hline
			Goal & Achieve[IncidentInfoProcessed\-WhenIncidentInfoRecorded] \\ \hline
			Agent & InformationProcessor \\ \hline
			État atteint & IncidentInfoKnown \\ \hline
			Évènement lié & IncidentInfoProcessing \\ \hline
			In & $i: IncidentInfo$ \\ \hline
			Out & $i: IncidentInfo$ \\ \hline
			Pre & $\exists c: Call, \exists j: Incident (c.about \rightarrow j \wedge c.reporting \rightarrow i)$ \\ \hline
			Post & $i.pos \neq null \wedge i.ambulanceKindNeeded \neq null $  \\
				 & La position (pos) contenue dans i correspond à la position (localisation) de i sous forme exploitable par le système. 
				  Le type d'ambulance nécessaire de i est calculé selon les règles données par le gouvernement et sur base des 
				  informations présentes dans i. \\ \hline
		\end{tabularx}
		\caption{L'opération \textit{Achieve[processIncidentInfo]}}\label{tab:op1}
	\end{table}

\newpage
\subsection{Enregistrement de la position}
	
	L'ambulance, élément du monde réel, se déplace avec le temps. Dans le système
	informatique, il nous faut naturellement refléter cette réalité au 
	sein du logiciel. L'opération \textit{Achieve[recordAccurateAmbulancePosition]}
	se charge d'enregistrer la position précise envoyée par l'ambulance
	dans le logiciel. La figure \ref{fig:op2} illustre
	le but opérationnalisé et le tableau \ref{tab:op2} synthétise l'opération.
	
	\vfill
	\insertfiguremargin{op2}{L'opération \textit{Achieve[recordAccurateAmbulancePosition]} enregistre l'opération. On peut voir le but correspondant sur la figure.}
	\vfill
	
	\begin{table}[!h] \centering
		\begin{tabularx}{\textwidth}{|l|X|} \hline
			Goal & Achieve[AccurateAmbulancePositionRecorded\\ &WhenAccurateAmbulancePositionSent] \\ \hline
			Agent & AmbulanceTracker \\ \hline
			Évènement lié & AccurateAmbulancePositionRecording \\ \hline
			In & $a: Ambulance$ \\ \hline
			Out & $b: AmbulanceInfo$ \\ \hline
			Pre & $\exists b: ambulanceInfo : a.id = b.id$ \\ \hline
			Post & $b.pos = a.pos$ \\ \hline
		\end{tabularx}
		\caption{L'opération \textit{Achieve[recordAccurateAmbulancePosition]}}\label{tab:op2}
	\end{table}
	
\newpage
\subsection{Choix de l'ambulance}
	
	Lors d'un incident, le système envoie une ambulance. Pour pouvoir envoyer
	la bonne ambulance, le logiciel doit nécessairement choisir quel
	ambulance envoyer. L'opération \textit{chooseAmbulance} s'occupe de choisir
	l'ambulance à envoyer. La figure \ref{fig:op3} contextualise le but
	opérationnalisé et le tableau \ref{tab:op3} synthétise l'opération.
	
	\vfill
	\insertfiguremargin{op3}{L'opération \textit{chooseAmbulance} opérationalise le but \textit{Achieve[AmbulanceChosen]}, le but est indiqué sur la figure, les autres étant plus claires.}
	\vfill
	
	\begin{table}[!h] \centering
		\begin{tabularx}{\textwidth}{|l|X|} \hline
			Goal & Achieve[AmbulanceChosen] \\ \hline
			Agent & InfoProcessor \\ \hline
			État atteint & BestAmbulanceChosen \\ \hline
			Évènement lié & BestAmbulanceChoice \\ \hline
			In & $i: IncidentInfo$ \\ \hline
			Out & $a: AmbulanceInfo$ \\ \hline
			Pre & $\exists a: AmbulanceInfo : \#a.mobilisation = 0 \wedge \#a.choice = 0$ \\ \hline
			Post & $\#a.choice = 1 \wedge \#i.choice = 1 \wedge i.choice \rightarrow a$ \\ \hline
		\end{tabularx}
		\caption{L'opération \textit{Achieve[chooseAmbulance]}}\label{tab:op3}
	\end{table}
	
\newpage
\subsection{Envoi de l'ordre de mobilisation}

	Une fois l'ambulance choisie, il est nécessaire de la mobiliser afin 
	qu'elle se rende sur place. L'opération \textit{Achieve[sendMobilizationOrder]}
	s'occupe d'envoyer l'ordre de mobilisation à l'ambulance.
	
	\insertfiguremargin{op4}{L'opération \textit{sendMobilizationOrder} opérationnalise le but \textit{Achieve[MobilizationOrderSentWhenAmbulanceChosen]}}
	
	\begin{table}[!h] \centering
		\begin{tabularx}{\textwidth}{|l|X|} \hline
			Goal & Achieve[MobilizationOrderSentWhenAmbulanceChosen] \\ \hline
			Agent & InfoProcessor \\ \hline
			État atteint & MobilizationOrderTransmitted \\ \hline
			Évènement lié & MobilizationOrderSending \\ \hline
			In & $a: AmbulanceInfo, i: IncidentInfo$ \\ \hline
			Out & $m: MobilizationOrder$ \\ \hline
			Pre & $\#a.choice = 1 \wedge a.choice \rightarrow i$ \\ \hline
			Post & $\exists m: MobilizationOrder : m.ambulanceId = a \wedge m.incidentId = i$ \\ \hline 
		\end{tabularx}
		\caption{L'opération \textit{sendMobilizationOrder}}\label{tab:op4}
	\end{table}
