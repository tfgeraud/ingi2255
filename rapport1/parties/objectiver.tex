Objectiver nous a été fort utile durant cette première partie du projet,
cependant quelques petites améliorations pourraient rendre son utilisation
nettement plus efficace.

\begin{itemize}
	\item Le format de sauvegarde en .xml met tout le fichier sur une seule
	ligne, S'il état correctement indenté, des outils tels que svn / git
	pourraient gérer automatiquement les conflits de version qui arrivent
	régulièrement lorsqu'on travaille à 8 sur un même fichier.
	
	\item Un "Search and Replace" dans le noms des buts et leurs définitions
	nous aurait fait gagner un temps précieux. 

	\item La manipulation des graphes pourrait être nettement améliorée,
	par exemple s'il était possible de déplacer un noeud parent et tous ses
	enfants en même temps, ou de "minimiser" un noeud et ses enfants. 

	\item Il serait pratique de pouvoir disposer de plusieurs instances d'un
	même but/objet/obstacle dans les diagrammes, cela permettrait de les 
	arranger plus clairement.

	\item Une vraie version linux ne serait pas de refus.

	\item L'outil d'export de graphes gagnerait à faire des svg corrects. Il
	est souvent plus rapide de créer directement le diagramme dans Inkscape ou
	Dia que de corriger les erreurs du fichier exporté. 

\end{itemize}



