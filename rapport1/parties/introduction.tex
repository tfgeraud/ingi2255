La méthode du \textit{Requirement Engineering} permet, entre autre, 
d'aborder des problématiques complexes afin d'en tirer l'essentiel et l'utile pour la 
conception de logiciels critiques. Dans le cadre du cours de génie logiciel, il
nous est demandé d'analyser les besoins tant fonctionnels que non fonctionnels
pour un système de gestion d'ambulance, de dériver les spécifications
logicielles nécessaire pour un tel système, de proposer une solution logiciel
de haute qualité et, enfin, de fournir un prototype d'un tel system.

Ce rapport présente la première partie du projet: l'analyse des besoins. 
Cette première version du rapport présente donc les besoins ainsi que les 
spécification logicielles (non formelle) pour notre système.
avons utilisé une modélisation par buts pour guider l'ensemble de l'analyse.
Le rapport est découpé en deux parties : 
\begin{itemize}
	\item la première explique la manière
		  dont nous avons construit notre modèle et illustre les liens inter-modèles
		  existants,
	\item La seconde partie liste les buts et les objets modélisés à l'aide
		  du logiciel Objectiver;
\end{itemize}
