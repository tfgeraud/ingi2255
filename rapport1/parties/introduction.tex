La méthode du \textit{Requirement Engineering} permet, entre autre, 
d'aborder des problématiques complexes afin d'en tirer l'essentiel et l'utile pour la 
conception de logiciels critiques. Dans le cadre du cours de génie logiciel, il
nous est demandé de concevoir un logiciel de gestion d'ambulances similaire
au logiciel utilisé à Londres il y a quelques années.

Ce rapport présente la première partie du projet: l'analyse des besoins. Nous
avons utilisé une modélisation par buts pour guider l'ensemble de l'analyse.
Le rapport est découpé en deux parties : la première explique la manière
dont nous avons construit notre modèle et illustre les liens inter-modèles
existants. La seconde partie liste les buts et les objets modélisés à l'aide
du logiciel Objectiver.
