
% Macro pour définir les modules de l'architecture, inspiré du miniprojet
%
% #1 Nom du module
% #2 Description du module
% #3 Justification du module proposé
% #4 Opérations réalisées
% #5 Opérations auxquelles le module contribue
\newcommand{\moddesc}[5]{
	\begin{table}[htdp]
	\begin{center}
		\begin{tabular}{|ll|}
		\hline \textbf{#1} & \\
		\hline Descr. & #2 \\
		 Why? & #3 \\
		 Perform & #4 \\
		  Contrib & #5 \\
		\hline 
		\end{tabular} 
		\caption{Module #1}
	\end{center}
	\label{defaulttable}
	\end{table}
}

% Macro pour définir les classes
% #1 Nom de la classe
% #2 Description des responsabilités de la classe
% #3 Définition des méthodes publiques de la classe (interface)
\newcommand{\classdesc}[3]{
	\begin{table}[htdp]
	\begin{center}
		\begin{tabular}{|ll|}
		\hline \textbf{#1} & \\
		\hline Descr. & #2 \\
		 Public methods & #3 \\
		\hline 
		\end{tabular} 
		\caption{Class #1}
	\end{center}
	\label{defaulttable}
	\end{table}	
}
