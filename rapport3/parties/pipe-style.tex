\subsection{Architecture \emph{Pipe and Filter}}
	Ce type d'architecture consiste en un flux de données,
	et une série de filtres qui les transforment séquentiellement,
	pour arriver au résultat désiré. Cette architecture est
	attractive car très simple, modulaire et facilement parallélisable,
	et nous avons donc cherché à savoir si elle pouvait s'adapter à
	notre problème.

	Un des principaux problèmes de ce type d'architecture est de
	s'accorder sur le type de données du flux. Ici un type s'imposait,
	à savoir l'incident. En effet tout dans notre systeme, tourne
	autour de celui ci: il démarre quand un nouvel incident est créé,
	et s'arrète lorsque l'incident est résolu. Chaque filtre modifierait
	donc la représentation interne de l'incident, mais surtout, s'assurerait
	que cette représentation s'accorde avec le monde externe.

	Cependant, au fur et à mesure que l'on considère cette architecture
	nous sommes forcés de nous en écarter. Premièrement, il n'est pas 
	possible de modifier l'état incrémentalement dans le module lui meme,
	les modifications de l'incident doivent etre atomiques.

	Ensuite, s'il est possible de réaliser une séquence de filtres
	indépendants, la réussite de chacun est dépendante du monde réel,
	et n'est donc pas garantie. Un échec d'un module requiert
	un retour au module précédent, voir tout au début. 
	
	Cela consiste donc finalement à implémenter la machine à état de 
	l'incident qui est notre architecture finale.

	Malgré le fait que cette architecture semblait assez éloignée
	de notre problème et qu'elle n'a finalement pas été retenue,
	elle aura largement influencé notre architecture finale, nous
	montrant ainsi qu'aucun shéma ne doit ètre écarté à priori.




