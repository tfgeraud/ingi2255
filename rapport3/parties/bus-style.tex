
\section{Architecture en bus}
	L'architecture en bus est une variante de l'architecture event-based dans
	laquelle chaque module génère et écoute des événements.  Contrairement au
	schéma général utilisant un "broker" qui centralise les "intêrets" des
	acteurs pour certains événements et notifie les acteurs interessés quand un
	événement survient, dans l'architecture en bus, tout les modules écoutent
	tous les événements et filtrent uniquement ceux qui les intéressent. Il est
	donc facile, dans cette dernière de modifier un module en y ajoutant ses
	intêrets pour certains événement sans devoir toucher aux autres modules.

	D'intuition cette architecture semble simple cependant elle n'est pas très
	adaptée pour notre système dans le sens où chaque module doit tenir un
	historique des états du système afin de raisonner correctement lorsqu'il
	reçoit un événement.  Un exemple simple est le cas où le système veut
	mobiliser une ambulance qui refuse la mobilisation, le système tente alors
	de mobiliser une deuxième ambulance qui refuse aussi. 

	Lorsque le système mobilisera une troisième ambulance, il devra faire
	attention à ne pas mobiliser la première qui à déjà refuser, ni la
	deuxième.
