Cette partie présente le plan de développement de notre logiciel ainsi que
la répartition du travail au sein du groupe.

Le travail est réparti en équipe de développeurs, ces développeurs sont des
équipes de deux ou trois personnes au maximum, le but étant de minimiser les 
interactions tout en conservant une relecture et une liberté d'implémentation
aux équipes.

Dans chacune des phases, l'équipe se voit assigner un ensemble de module à 
écrire et à tester, sur base des tests black-box précédemment conçu. Ces tests
seront rédigé à l'aide JUnit et serviront également de tests de régressions.
Les tests seront écrits à l'aide de JUnit 4 et le code java sera écrit
en code compatible Java 6.

\section{Phase 1 : La gestion de l'information}
Cette phase permet de mettre en place tous les objets qui seront utilisé
par le reste de l'application. Ces objets seront stocké, dans un premier
temps, pour la durée de l'exécution du logiciel.

Les modules concernés par cette phase sont, pour la
partie système: Incident, Map, Ambulance et pour la partie simulateur : SimObjects, Map.
La répartition du travail pour 
les différents module et pour les équipes est décrite ci-dessous:

\noindent\begin{tabularx}{\textwidth}{|l|l|X|}
\hline
Module			& 	Équipe de développement & Équipe de test\\
\hline
Incident	&	Team A 	&	Team B	\\
Map 		&	Team B	&	Team A	\\
Ambulance	&	Team A  &	Team B	\\
SimObjects	&	Team A  &	Team B	\\
Map			&	Team B  &	Team A	\\
\hline
\end{tabularx}

Fin de la première phase: 27 novembre.

\section{Phase 2 : Communication}
Cette phase met en place la communication entre les deux mondes.

Les modules concernés par cette seconde phase sont les suivants, pour
le système : Communicator, Broker et pour le simulateur : Communication, CallSimul, AVLS et MDT.
À nouveau, la répartition est présentée dans le tableau suivant:

\noindent\begin{tabularx}{\textwidth}{|l|l|X|}
\hline
Module			& 	Équipe de développement & Équipe de test\\
\hline
Communicator	&	Team B 	&	Team A\\
Broker			&	Team B	&	Team A\\
Communication 	&	Team B	&	Team A	\\
CallSimul		&	Team A	&	Team B	\\
AVLS			&	Team A	&	Team B	\\
MDT				&	Team A	&	Team B	\\
\hline
\end{tabularx}

Fin de la seconde phase: 2 décembre.

\section{Phase 3 : Le coeur}
Cette phase va s'appuyer sur les phases précédentes afin de les exploiter
et de faire en sorte que le logiciel fasse ce pourquoi ce dernier a été conçu.

Les modules concernés sont les suivants: pour le système: IncidentProcessor,
AmbulanceChooser, Mobilizer et Resolver, pour le simulateur: Simulator,
Scenario et FileScenario.
À nouveau, la répartition est présentée dans le tableau suivant:

\noindent\begin{tabularx}{\textwidth}{|l|l|X|}
\hline
Module			& 	Équipe de développement & Équipe de test\\
\hline
IncidentProcessor	&	Team A 	&	Team B\\
AmbulanceChooser	&	Team B	&	Team A\\
Mobilizer 			&	Team A	&	Team B	\\
Resolver			&	Team A	&	Team B	\\
Simulator			&	Team B	&	Team A	\\
Scenario			&	Team A	&	Team B	\\
FileScenario		&	Team B	&	Team A	\\
\hline
\end{tabularx}

Fin de la seconde phase: 11 décembre.

\section{Phase 4 : L'enrichissement}
Cette phase est l'occasion d'ajouter des modules à notre architecture afin de 
proposer une plus grand nombre de fonctionnalité.

En fonction du temps et des affinités des équipes, il sera possible d'implémenter
certaines de ces fonctionnalités.

Parmis ces fonctionnalités, nous avons:
\begin{itemize}
	\item Utilisation d'une base de donnée relationnelle pour sauvegarder les 
		  données de manière pérenne.
	\item L'ajout d'un module de statistique du coté du simulateur
	\item L'ajout d'un module de statistique du coté du système
	\item L'ajout d'un module s'occupant du placement des ambulances
		  afin de maximiser la couverture géographique
	\item L'utilisation d'une carte plus complexe, pour louvain-la-neuve par
		  exemple.
	\item Le développement de canaux de communication supplémentaire (Radio, 
	      téléphonne, etc.)
	\item L'ajout de scénario probabiliste
	\item L'ajout de scénario généré manuellement
	\item ...
\end{itemize}
