Cette section décrit les différents messages qui sont échangés entre le
système logiciel et l'environement, simulé ou non. Ces évènement sont échangé
aux travers des différentes interfaces de communications possible entre 
les deux lieux : l'AVLS, le MDT et l'interface d'entrée des informations
pour les incidents.

\begin{description}
	\item [Incident:] L'événement incident est envoyé à partir du simulateur vers le système et comporte les informations suivantes: 		
	l'âge de la victime, enceinte ou non, la localisation de l'incident (typiquement l'adresse), et une description de l'incident.
	\item [AVLSMessage:]L'événement AVLSMessage est envoyé à partir du simulateur vers le système et comporte la posisition (coordonnées 					géographique) et l'identifiant d'un ambulance.
	\item [MDTMessage:] L'événement MDTMessage peut provenir soit du système soit du simulateur. Il y a plusieurs type de MDTMessage
\end{description}	


%MDTMessage du système vers le simulateur
\begin{table}[!h]
	\centering
	\begin{tabularx}{\marginparsep+\marginparwidth+\marginparpush+\textwidth}{|l|X|X|}
		\hline
		\textbf{nom du message} & \textbf{arguments} & \textbf{description} \\ \hline
		mobilisationOrder & incidentID, incidentPosition, ambulanceID & Le message de mobilisation est envoyé par le système à une ambulance (AmbulcanceID) afin de la mobiliser pour un incident (incidentID) qui à lieu à la position (incidentPosition) \\ \hline
		demobilisationOrder & incidentID, incidentPosition, ambulanceID & Le message demobilisationOrder est envoyé par le système à une ambulance (AmbulanceID) afin de la démobiliser pour l'incident (incidentId) se trouvant à la position (incidentPosition) \\ \hline
	\end{tabularx} 
	\caption{MDTMessage du système vers le simulateur}
	\label{tab:MDTMessage}
\end{table}

%MTDMessage du simulateur vers le système
\begin{table}[!h]
	\centering
	\begin{tabularx}{\marginparsep+\marginparwidth+\marginparpush+\textwidth}{|l|X|X|}
		\hline
		\textbf{nom du message} & \textbf{arguments du message} & \textbf{description du message} \\ \hline
		mobilisationConfirmation & incidentID, ambulanceID, un booleen yes/no & Un message de confirmation est envoyé par l'ambulance (ambulanceID) pour accepter (yes) ou refuser (no) l'ordre de mobilisation concernant l'incident (incidentID) \\ \hline 
		ambulanceBroken & ambulanceID & Message envoyé par l'ambulance (ambulanceID) pour signaler qu'elle est en panne \\ \hline
		ambulanceRepaired & ambulanceID & Message envoyé par l'ambulance (ambulanceID) pour signaler qu'elle est réparée \\ \hline
		obstacle & position & Un message Obstacle avec une position en argument est envoyé par une ambulance au système pour signaler qu'il existe une obstacle à cette position \\ \hline
		incidentCancelled & incidentID, ambulanceID & Un message incidentCancelled est envoyé par l'ambulance (ambulanceID) au système pour signaler que l'intervention pour l'incident (incidentID) est annulée \\ \hline
		incidentResolved & incidentID, ambulanceID & Un message incidentResolved est envoyé par l' ambulance (ambulanceID) au système pour signaler que l'intervention pour l'incident (incidentID) est résolue \\ \hline
		destinationUnreachable & ambulanceID, incidentID & Une message destinationUnreachable est envoyé par l'ambulance (ambulanceID) au système pour signaler que la position de l'incident (incidentID) est inaccessible \\ \hline
	\end{tabularx} 
	\caption{MDTMessage du simulateur vers le système}
	\label{tab:MDTMessage}
\end{table}
