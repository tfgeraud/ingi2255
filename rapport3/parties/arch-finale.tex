\subsection{Présentation}
	Afin de présenter l'architecture, nous allons effectuer
	un parcours de celle ci pour le cas idéal de résolution
	d'incident.

	On commence à l'InfoGUI qui est l'interface graphique que
	l'opérateur téléphonique utilise pour entrer les données
	caractérisant l'incident.

	Ces données sont transmises au Broker. Celui possède deux
	fonctions. La première, architecturale, est de présenter
	une interface unique aux différents système d'introduction
	de données : Interface Utilisateur, simulateur, ou encore 
	d'autres logiciels de gestion d'hopitaux ou de pompiers par
	exemple.
	
	L'autre role, plus fonctionnel, du broker est d'instancier
	un Incident, et de transmettre
	une référence vers celui ci à l'InfoProcessor.

	L'InfoProcesseur est le chef d'orchestre de notre systeme.
	Il se charge d'abord de compléter la représentation de
	l'Incident en évaluant le type d'ambulance requise
	et les coordonnées GPS de la destination. pour cela
	il s'aidera du module Map et Incident.

	Ensuite, L'InfoProcessor va appeler différent modules
	afin de faire avancer état par état la résolution de
	l'Incident. 

	En premier il devra choisir l'Abulance grace à AmbulanceChooser
	celui ci fera pour cela appel au module Map pour
	évaluer les distances, et au module Ambulance pour savoir
	lesquelles sont libres. 

	Une fois l'ambulance choisie,le Mobilizer se chargera d'envoyer
	l'ordre de mobilisation. Il s'aidera du Communicator pour 
	envoyer l'ordre et récuperer la confirmation. 

	Le module Resolver permettra de gérer les events intervenant
	entre le moment ou l'ambulance est mobilisée et le moment
	ou elle à réussi ou échoué à résoudre le problème.

	Le module Mobilizer est ensuite utilisé à nouveau pour démobiliser
	l'ambulance ayant terminé sa mission. 

	Tout au long de cette opération, le module Communicator se
	charge de recevoir les events provenant du monde extérieur
	et de mettre à jour la base de données des ambulances et de
	la carte. Il s'assure aussi de l'intégrité des données recues,
	et d'envoyer au modules de résolutions des events qui 
	peuvent interrompre ceux ci.

\subsection{Gestion des Obstacles dans l'architecture}
	L'architecture modélisant la machine à état de l'Incident, 
	la résolution d'obstacles est aisée et ne nécessite généralement
	que la modification que de quelques modules.
	\begin{itemize}
		\item Communicator afin de prendre en compte un nouveau type
		 	d'event correspondant à l'obstacle si nécessaire.
		\item le module ambulance, map ou Incident afin d'y ajouter
			les propriétés supplémentaires nécessaires.
		\item le module chargé du changement d'état de l'Incident 
		      concerné par l'obstacle.
		\item Si les changements outrepassent le secret du module
			précédent, il faudra modifier IncidentProcessor.
	\end{itemize}
	
	Par exemple la résolution de l'obstacle consistant en une duplication
	des incidents consiste à ajouter à Incident une opération permettant
	de détecter un duplicata, et de modifier IncidentProcessor pour
	utilisant et d'arrèter la résolution dans de tels cas.

\subsection{Evolutions de l'Architecture}







