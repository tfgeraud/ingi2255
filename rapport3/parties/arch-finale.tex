\subsection{Définition des modules}
Les définitions sont présentées dans le tableau \ref{tab:defmodule}.

\begin{table}[!h]
\begin{tabularx}{\marginparsep+\marginparwidth+\marginparpush+\textwidth}{l|X}
Module & Définition \\ \hline
CallSimul & Ce module est reponsable de la notification des incidents en 
provenance du simulateur. \\
InfoGUI & Ce module est responsable de l'interface graphique permettant 
l'ajout d'incident dans le système. \\
Broker & Ce module est responsable de fournir une API fixe pour l'ajout
d'incident au sein du système.\\
IncidentProcessor & Ce module est reponsable du suivi des incidents, 
il s'occupera de faire en sorte que l'incident soit traité corectement
et dans les délais. \\
AmbulanceChoser & Ce module est reponsable du choix de l'ambulance. \\
Mobilizer &  Ce module est responsable de la mobilisation virtuelle et
concrète de l'ambulance \\
Resolver &  Ce module est responsable de la correcte terminaison d'un
incident. \\
Incident & Ce module est responsable de la gestion des incidents au sein
du système. \\
Map & Ce module est responsable de la gestion de la carte et donner
les informations appropriés aux autres modules. \\
Ambulance &  Ce module est reponsable de la gestion des ambulances  et
de l'information qui y est lié au sein du système. \\
Communicator &  Ce module est reponsable de la gestion correcte
des communications à partir du système. \\
Avls & Ce module est responsable de l'envoi correct de la position
de l'ambulance. \\
MDT & Ce module est reponsable de la communication entre les équipages
de l'ambulance et le système. \\
\end{tabularx}
\caption{Définition des différents modules utilisé au sein du système.}\label{tab:defmodule}
\end{table}

\subsection{Lien entre les modules et agencement}
	Afin de présenter l'architecture, nous allons effectuer
	un parcours de celle ci pour le cas idéal de résolution
	d'incident.

	Nous commençons par l'InfoGUI qui est l'interface graphique que
	l'opérateur téléphonique utilise pour entrer les données
	caractérisant l'incident.

	Ces données sont transmises au Broker. Celui possède deux
	fonctions. La première, architecturale, est de présenter
	une interface unique aux différents système d'introduction
	de données : Interface Utilisateur, simulateur, ou encore 
	d'autres logiciels de gestion d'hopitaux ou de pompiers par
	exemple.
	
	L'autre rôle, plus fonctionnel, du broker est d'instancier
	un Incident, et de transmettre
	une référence vers celui ci à l'InfoProcessor.

	L'InfoProcesseur est le chef d'orchestre de notre système.
	Il se charge d'abord de compléter la représentation de
	l'Incident en évaluant le type d'ambulance requise
	et les coordonnées GPS de la destination. pour cela
	il s'aidera du module Map et Incident.

	Ensuite, L'InfoProcessor va appeler différent modules
	afin de faire avancer état par état la résolution de
	l'Incident. 

	En premier il devra choisir l'ambulance grâce à AmbulanceChooser
	celui ci fera pour cela appel au module Map pour
	évaluer les distances, et au module Ambulance pour savoir
	lesquelles sont libres. 

	Une fois l'ambulance choisie, le Mobilizer se chargera d'envoyer
	l'ordre de mobilisation. Il s'aidera du Communicator pour 
	envoyer l'ordre et récuperer la confirmation. 

	Le module Resolver permettra de gérer les évènements intervenant
	entre le moment ou l'ambulance est mobilisée et le moment
	ou elle à réussi ou échoué à résoudre l'incident.

	Le module Mobilizer est ensuite utilisé à nouveau pour démobiliser
	l'ambulance ayant terminé sa mission. 

	Tout au long de cette opération, le module Communicator se
	charge de recevoir les events provenant du monde extérieur
	et de mettre à jour la base de données des ambulances et de
	la carte. Il s'assure aussi de l'intégrité des données recues,
	et d'envoyer au modules de résolutions des events qui 
	peuvent interrompre ceux ci.

\subsection{Gestion des obstacles dans l'architecture}
	L'architecture modélisant la machine à état de l'Incident, 
	la résolution d'obstacles est aisée et ne nécessite généralement
	que la modification que de quelques modules.
	\begin{itemize}
		\item Communicator afin de prendre en compte un nouveau type
		 	d'event correspondant à l'obstacle si nécessaire.
		\item le module ambulance, map ou Incident afin d'y ajouter
			les propriétés supplémentaires nécessaires.
		\item le module chargé du changement d'état de l'Incident 
		      concerné par l'obstacle.
		\item Si les changements outrepassent le secret du module
			précédent, il faudra modifier IncidentProcessor.
	\end{itemize}
	
	Par exemple la résolution de l'obstacle consistant en une duplication
	des incidents consiste à ajouter à Incident une opération permettant
	de détecter un duplicata, et de modifier IncidentProcessor pour
	l'utiliser et arrèter la résolution dans de tels cas.

\subsection{Evolutions du système et de l'architecture}
	Nous avons énuméré une série d'évolutions du système et avons
	identifié les modules devant etres modifiés ou ajoutés.
	\subsubsection{Nouveaux types de véhicules}
		\begin{itemize}
			\item Modification d'Ambulance: nouvelle catégorie,
			\item Modification d'AmbulanceChooser: le calcul de la plus courte
				distance peut etre différent avec le nouveau type, et il
				y a une nouvelle catégorie de véhicule à choisir,
			\item Eventuelle modification de Map pour ajouter une distance 
			      spécifique au véhicule ( par exemple à vol d'oiseau pour un
			      hélicoptère ) ;
		\end{itemize}
	\subsubsection{Incident demandant l'intervention de plusieurs ambulances}
		\begin{itemize}
			\item Modification d'InfoProcessor : choisit plusieurs ambulances;
		\end{itemize}
	\subsubsection{Intégration aux systèmes de Police, d'Hopital ou Pompiers }
		\begin{itemize}
			\item Modification de Broker : Nouveaux canaux de création d'incidents,
			\item Modification de Communicator : Nouveaux canaux d'écoute pour obstacles;
		\end{itemize}
	\subsubsection{Annulation d'un incident}
		\begin{itemize}
			\item Modification de Broker et Communicator : nouveaux events de
			 	demande d'annulation,
			\item Modification d'InfoProcessor : écoute de ces events, et
				gestion de ceux-ci;
		\end{itemize}
	\subsubsection{Ajout d'un canal de communication}
		\begin{itemize}
			\item Modification de Communicator : écoute et envoi dans ce nouveau canal;
		\end{itemize}
	\subsubsection{Utilisation d'une base de données afin de sauvegarder l'état du système}
		\begin{itemize}
			\item Ajout d'un module Database,
			\item Modification d'Incident,Ambulance,Map pour faire usage de celle-ci;
		\end{itemize}
	\subsubsection{Utilisation de données géographiques plus complexes.}
		\begin{itemize}
			\item Modification du module Map;
		\end{itemize}
	\subsubsection{Réalisation de statistiques sur le comportement du système}
		\begin{itemize}
			\item Ajout d'un module Statistique qui utilise Incident,Ambulance et Map;
		\end{itemize}
	\subsubsection{Suivi de la progression de l'incident par les témoins.}
		\begin{itemize}
			\item Ajout d'un nouveau module FollowUpGUI utilisant Map,Incident et Ambulance
				pour permettre à l'opérateur téléphonique d'obtenir des informations
				de suivi sur l'incident;
		\end{itemize}
		
