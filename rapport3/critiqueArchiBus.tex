%Critique des architectures possible
\section{Architecture en bus}
  \paragraph{}
  L'architecture en bus est une variante de l'architecture event-based dans laquelle chaque module g�n�re/�coute des �v�nements.
  Contrairement au sch�ma g�n�ral utilisant un "broker" qui centralise les "int�rets" des acteurs pour certains �v�nements et notifie les acteurs interess�s quand un �v�nement survient,
  dans l'architecture en bus, tout les modules �coutent tous les �v�nements et filtrent uniquement ceux qui les int�ressent. Il est donc facile, dans cette derni�re de modifier un module en y ajoutant ses int�rets pour certains �v�nement sans devoir toucher aux autres modules.

  \paragraph{}
  D'intuition cette architecture semble simple cependant elle n'est pas tr�s adapt�e pour notre syst�me dans le sens o� chaque module doit tenir un historique des �tats du syst�me afin de raisonner correctement lorsqu'il re�oit un �v�nement.
  Un exemple simple est le cas o� le syst�me veut mobiliser une ambulance qui refuse la mobilisation, le syst�me tente alors de mobiliser une deuxi�me ambulance qui refuse aussi. 
  Lorsque le syst�me mobilisera une troisi�me ambulance, il devra faire attention � ne pas mobiliser la premi�re qui � d�j� refuser, ni la deuxi�me.
  
  