\documentclass{report}

\usepackage[utf8]{inputenc}
\usepackage{style-report}

% L'en-tête et la mise en page sera différente
% Merci de ne pas faire de mise en page dans le document !
\title{London Ambulance System}
\subtitle{Architecture logicielle}
\author{\normalsize{Groupe 2}\\
\footnotesize{
Simon Busard, \\
Antoine Cailliau, \\
Laurent Champon,\\
Erick Lavoie, \\
Quentin Pirmez,\\
Frederic Van der Essen, \\
Géraud Talla Fotsing}}

\date{\today}

% Pour modifier la géométrie de la page
\usepackage{geometry}
\geometry{a4paper,left=3cm,right=3cm,top=2.5cm,bottom=4cm}

\newlength{\realtextwidth}
\setlength{\realtextwidth}{\textwidth}
\newcommand{\insertfiguremargin}[2]{
	\begin{figure}[!h]
	\noindent\begin{minipage}[!h]{\marginparsep+\marginparwidth+\marginparpush+\textwidth}
		\begin{minipage}[t]{\realtextwidth}
			\vspace{0pt}
			\includegraphics[width=\realtextwidth]{images/#1.png}
		\end{minipage}
		\hspace{\marginparsep}
		\begin{minipage}[t]{\marginparwidth+\marginparpush}
			\vspace{0pt}
			\setcaptionwidth{\marginparwidth+\marginparpush}
			\caption{#2}\label{fig:#1}
			\setcaptionwidth{0.9\realtextwidth}
		\end{minipage}		
	\end{minipage}
	\end{figure}

}

\setlength{\realtextwidth}{\textwidth}
\newcommand{\insertfigure}[2]{
	\begin{figure}[!h]
	\noindent\begin{minipage}[!h]{\marginparsep+\marginparwidth+\marginparpush+\textwidth}
		\begin{minipage}[t]{\textwidth}
			\vspace{0pt}
			\includegraphics[width=\textwidth]{images/#1.png}
		\end{minipage}
		
		\hspace{\realtextwidth}
		\hspace{\marginparsep}
		\begin{minipage}[t]{\marginparwidth+\marginparpush}
			\vspace{0.5cm}
			\setcaptionwidth{\marginparwidth+\marginparpush}
			\caption{#2}\label{fig:#1}
			\setcaptionwidth{0.9\realtextwidth}
		\end{minipage}		
	\end{minipage}
	\end{figure}

}

\begin{document}

\setlength{\parskip}{1em}
\startdocument

\maketitle
\setcounter{tocdepth}{1}
\tableofcontents

\chapter{Architecture logique}
	
	Cette section présente les alternatives d'architecture que nous
	avons exploré. Ensuite, nous détaillons l'architecture que nous
	avons finalement choisie.

	\section{Architectures évaluées }
		
		Cette section présente les deux autres alternatives que nous 
		avons exploré, à savoir l'architecture \textit{pipe} et l'architecture
		\textit{bus}.
	
		\subsection{Architecture \emph{Pipe and Filter}}
	Ce type d'architecture consiste en un flux de données,
	et une série de filtres qui les transforment séquentiellement,
	pour arriver au résultat désiré. Cette architecture est
	attractive car très simple, modulaire et facilement parallélisable,
	et nous avons donc cherché à savoir si elle pouvait s'adapter à
	notre problème.

	Un des principaux problèmes de ce type d'architecture est de
	s'accorder sur le type de données du flux. Ici un type s'imposait,
	à savoir l'incident. En effet tout dans notre systeme, tourne
	autour de celui ci: il démarre quand un nouvel incident est créé,
	et s'arrète lorsque l'incident est résolu. Chaque filtre modifierait
	donc la représentation interne de l'incident, mais surtout, s'assurerait
	que cette représentation s'accorde avec le monde externe.

	Cependant, au fur et à mesure que l'on considère cette architecture
	nous sommes forcés de nous en écarter. Premièrement, il n'est pas 
	possible de modifier l'état incrémentalement dans le module lui meme,
	les modifications de l'incident doivent etre atomiques.

	Ensuite, s'il est possible de réaliser une séquence de filtres
	indépendants, la réussite de chacun est dépendante du monde réel,
	et n'est donc pas garantie. Un échec d'un module requiert
	un retour au module précédent, voir tout au début. 
	
	Cela consiste donc finalement à implémenter la machine à état de 
	l'incident qui est notre architecture finale.

	Malgré le fait que cette architecture semblait assez éloignée
	de notre problème et qu'elle n'a finalement pas été retenue,
	elle aura largement influencé notre architecture finale, nous
	montrant ainsi qu'aucun shéma ne doit ètre écarté à priori.





		
\section{Architecture en bus}
	L'architecture en bus est une variante de l'architecture event-based dans
	laquelle chaque module génère et écoute des événements.  Contrairement au
	schéma général utilisant un "broker" qui centralise les "intêrets" des
	acteurs pour certains événements et notifie les acteurs interessés quand un
	événement survient, dans l'architecture en bus, tout les modules écoutent
	tous les événements et filtrent uniquement ceux qui les intéressent. Il est
	donc facile, dans cette dernière de modifier un module en y ajoutant ses
	intêrets pour certains événement sans devoir toucher aux autres modules.

	D'intuition cette architecture semble simple cependant elle n'est pas très
	adaptée pour notre système dans le sens où chaque module doit tenir un
	historique des états du système afin de raisonner correctement lorsqu'il
	reçoit un événement.  Un exemple simple est le cas où le système veut
	mobiliser une ambulance qui refuse la mobilisation, le système tente alors
	de mobiliser une deuxième ambulance qui refuse aussi. 

	Lorsque le système mobilisera une troisième ambulance, il devra faire
	attention à ne pas mobiliser la première qui à déjà refuser, ni la
	deuxième.

	
	\section{Architecture finale du système}
		Cette section présente l'architecture du système de gestion d'ambulance.
		La section 1.3.1 présente les définitions des différents modules, 
		la section 1.3.2 présente les liens entre ces différents modules, 
		la section 1.3.3 présente la gestion des divers obstacles au sein 
		de notre architecture, la section 1.3.4 présente les évolutions du 
		système, la section 1.3.5 justifie l'utilisation des liens USE au sein
		de l'architecture. 
	
		\insertfiguremargin{logic-sys}{Architecture logique pour la partie du 
		système.}
		\subsection{Présentation}
	Afin de présenter l'architecture, nous allons effectuer
	un parcours de celle ci pour le cas idéal de résolution
	d'incident.

	On commence à l'InfoGUI qui est l'interface graphique que
	l'opérateur téléphonique utilise pour entrer les données
	caractérisant l'incident.

	Ces données sont transmises au Broker. Celui possède deux
	fonctions. La première, architecturale, est de présenter
	une interface unique aux différents système d'introduction
	de données : Interface Utilisateur, simulateur, ou encore 
	d'autres logiciels de gestion d'hopitaux ou de pompiers par
	exemple.
	
	L'autre role, plus fonctionnel, du broker est d'instancier
	un Incident, et de transmettre
	une référence vers celui ci à l'InfoProcessor.

	L'InfoProcesseur est le chef d'orchestre de notre systeme.
	Il se charge d'abord de compléter la représentation de
	l'Incident en évaluant le type d'ambulance requise
	et les coordonnées GPS de la destination. pour cela
	il s'aidera du module Map et Incident.

	Ensuite, L'InfoProcessor va appeler différent modules
	afin de faire avancer état par état la résolution de
	l'Incident. 

	En premier il devra choisir l'Abulance grace à AmbulanceChooser
	celui ci fera pour cela appel au module Map pour
	évaluer les distances, et au module Ambulance pour savoir
	lesquelles sont libres. 

	Une fois l'ambulance choisie,le Mobilizer se chargera d'envoyer
	l'ordre de mobilisation. Il s'aidera du Communicator pour 
	envoyer l'ordre et récuperer la confirmation. 

	Le module Resolver permettra de gérer les events intervenant
	entre le moment ou l'ambulance est mobilisée et le moment
	ou elle à réussi ou échoué à résoudre le problème.

	Le module Mobilizer est ensuite utilisé à nouveau pour démobiliser
	l'ambulance ayant terminé sa mission. 

	Tout au long de cette opération, le module Communicator se
	charge de recevoir les events provenant du monde extérieur
	et de mettre à jour la base de données des ambulances et de
	la carte. Il s'assure aussi de l'intégrité des données recues,
	et d'envoyer au modules de résolutions des events qui 
	peuvent interrompre ceux ci.

\subsection{Gestion des Obstacles dans l'architecture}
	L'architecture modélisant la machine à état de l'Incident, 
	la résolution d'obstacles est aisée et ne nécessite généralement
	que la modification que de quelques modules.
	\begin{itemize}
		\item Communicator afin de prendre en compte un nouveau type
		 	d'event correspondant à l'obstacle si nécessaire.
		\item le module ambulance, map ou Incident afin d'y ajouter
			les propriétés supplémentaires nécessaires.
		\item le module chargé du changement d'état de l'Incident 
		      concerné par l'obstacle.
		\item Si les changements outrepassent le secret du module
			précédent, il faudra modifier IncidentProcessor.
	\end{itemize}
	
	Par exemple la résolution de l'obstacle consistant en une duplication
	des incidents consiste à ajouter à Incident une opération permettant
	de détecter un duplicata, et de modifier IncidentProcessor pour
	l'utiliser et arrèter la résolution dans de tels cas.

\subsection{Evolutions du Systeme et de l'architecture}
	Nous avons énuméré une série d'évolutions du système et avons
	identifié les modules devant etres modifiés ou ajoutés.
	\subsubsection{Nouveaux types de véhicules}
		\begin{itemize}
			\item Modification d'Ambulance: nouvelle catégorie.
			\item Modification d'AmbulanceChooser: le calcul de la plus courte
				distance peut etre différent avec le nouveau type, et il
				y a une nouvelle catégorie de véhicule à choisir.
			\item Eventuelle modification de Map pour ajouter une distance 
			      spécifique au véhicule ( par exemple à vol d'oiseau pour un
			      hélicoptère ) 
		\end{itemize}
	\subsubsection{Incident demandant l'intervention de plusieurs ambulances}
		\begin{itemize}
			\item Modification d'InfoProcessor : choisit plusieurs ambulances.
		\end{itemize}
	\subsubsection{Intégration aux systèmes de Police, d'Hopital ou Pompiers }
		\begin{itemize}
			\item Modification de Broker : Nouveaux canaux de création d'incidents.
			\item Modification de Communicator : Nouveaux canaux d'écoute pour obstacles.
		\end{itemize}
	\subsubsection{Annulation d'un incident}
		\begin{itemize}
			\item Modification de Broker et Communicator : nouveaux events de
			 	demande d'annulation.
			\item Modification d'InfoProcessor : écoute de ces events, et
				gestion de ceux-ci.
		\end{itemize}
	\subsubsection{Ajout d'un canal de communication}
		\begin{itemize}
			\item Modification de Communicator : écoute et envoi dans ce nouveau canal.
		\end{itemize}
	\subsubsection{Utilisation d'une base de données afin de sauvegarder l'état du système}
		\begin{itemize}
			\item Ajout d'un module DataBase,
			\item Modification d'Incident,Ambulance,Map pour faire usage de celle-ci.
		\end{itemize}
	\subsubsection{Utilisation de données géographiques plus complexes.}
		\begin{itemize}
			\item Modification du module Map
		\end{itemize}
	\subsubsection{Réalisation de statistiques sur le comportement du système}
		\begin{itemize}
			\item Ajout d'un module Statistique qui utilise Incident,Ambulance et Map
		\end{itemize}
	\subsubsection{Suivi de la progression de l'incident par les témoins.}
		\begin{itemize}
			\item Ajout d'un nouveau module FollowUpGUI utilisant Map,Incident et Ambulance
				pour permettre à l'opérateur téléphonique d'obtenir des informations
				de suivi sur l'incident. 
		\end{itemize}
	
	
		
	








		\insertfiguremargin{logic-sys}{Architecture logique pour la partie du 
système.}

\subsection*{Justification des liens USE}
Chaque lien USE dans le schéma peut être justifié. Le tableau suivant
synthétise la justification des différents liens.

\begin{table}[!h]
\begin{tabularx}{\marginparsep+\marginparwidth+\marginparpush+\textwidth}{l|l|X}
Source & Destination & Justification \\ \hline
CallSimul et InfoGUI & Broker & L'implémentation de l'interface d'ajout d'un
incident ou l'interface d'ajout automatique d'un incident dépends de la 
spécification externe du Broker. Ce dernier étant responsable de 
donner une API fixe pour l'ajout d'incident de la part du monde extérieur. \\
Broker & IncidentProcessor & L'implémentation du broker dépends de la 
spécification externe de l'incidentProcessor. Le broker est responsable
de l'appel et du lancement de la procédure de gestion des incidents. \\
IncidentProcessor & AmbulanceChooser & L'implémentation correcte de 
l'IncidentProcessor dépends de la spécification externe de l'AmbulanceChooser. Si le
module ne retourne pas l'ambulance choisie alors le reste de l'implémentation
du traitement de l'incident peut ne plus être correcte. \\
IncidentProcessor & Mobilizer & L'implémentation correcte de 
l'IncidentProcessor dépends de la spécification externe du Mobilizer. Si
l'ambulance n'est pas correctement mobilisée alors le reste de la 
procédure de traitement de l'incident ne sera peut-être pas correcte.\\
IncidentProcessor & Resolver & L'implémentation correcte de 
l'IncidentProcessor dépends de la spécification externe du Resolver. Si
l'incident n'est pas correctement clot alors le reste de la 
procédure de traitement de l'incident ne sera peut-être pas correcte. \\
* & Ambulance & L'implémentation des différents modules dépendra directement
des spécifications externe du module Ambulance, ce dernier étant 
reponsable de la gestion de l'information utilisée à travers le système. \\
* & Map & L'implémentation des différents modules dépendra directement
des spécifications externe du module Map, ce dernier étant 
reponsable de la gestion de l'information utilisée à travers le système. \\
* & Incident & L'implémentation des différents modules dépendra directement
des spécifications externe du module Incident, ce dernier étant 
reponsable de la gestion de l'information utilisée à travers le système. \\
Mobilizer & Communicator & L'implémentation correcte du Mobilizer dépends
directement de la spécification externe du Communicator. En effet, si 
le Communicator n'envoi pas correctement les informations, le système 
peut ne plus fonctionner correctement. \\
Communicator & AVLS et MDT & L'implémentation du Communicator dépends 
directement de la spécification externe de l'AVLS et du MDT. En effet, si 
le comportement de l'AVLS ou du MDT change alors l'implémentation du Communicator
peut ne plus être correcte.
\end{tabularx}
\end{table}


\insertfiguremargin{phys-sys}{Architecture physique pour la partie du 
système.}

		
	\section{Architecture finale du simulateur}	
		La section 1.4.1 présente la définition
		des modules du simulateur. La section 1.4.2 présente la 
		justification des liens USE au sein du simulateur.	
		
		\begin{table}[!h]
\begin{tabularx}{\marginparsep+\marginparwidth+\marginparpush+\textwidth}{l|X}
Module & Définition \\ \hline
GUI & Ce module est responsable de l'interface graphique permettant de modifier les paramètres du simulateur \\
Simulator & Ce module est responsable de la gestion de la simulation d'un incident à partir des informations recues d'un scénario  \\
SimObjects & Ce module est responsable la gestion des objets de simulation au sein de l'environnement simulé \\
Ambulance & Ce module est responsable de la gestion des ambulances et de l'information qui y est lié au sein de l'environnement simulé \\
Incident & Ce module est reponsable de la gestion des incidents et de l'information qui y est lié au sein de l'environnement simulé \\
Map &  Ce module est responsable de la gestion de la carte de l'environnement simulé \\
Scenario & Ce module est responsable de transformer les différents format de simulations en format unique compréhensible du simulateur   \\
FileScenario & Ce module est responsable de la creation de simulations prédéfinies contenues dans un fichier.  \\
\end{tabularx}
\caption{Définition des différents modules utilisé au sein du système.}\label{tab:defmodule}
\end{table}
		\subsection{Présentation de la structure}

	L'architecture proposée pour le simulateur implémente une architecture
	basée sur MVC.  En l'occurence, le modèle correspond aux objets de
	simulation, le contrôleur au simulateur et la vue correspond au GUI.
	Additionnellement, la définition du scénario à exécuter est définie dans le
	Scénario et la communication est regroupée sous communication.
	L'utilisation du patron de conception \textit{Observer} est utilisée sur
	les objets de simulation et les canaux de communication pour permettre un
	maximum de flexibilité dans la connection des différents objets et
	faciliter la propagation des évènements aux travers du simulateur.

	La simulation est synchrone au sens où il y a un temps global défini pour
	l'exécution de chacun des objets.  Tous se synchronisent sur le step initié
	par le simulateur.  Le choix d'une simulation synchrone a été préféré à une
	simulation asynchrone car il est plus facile de contrôler l'évolution de la
	simulation en présence de multi-processus.  Il est alors possible de
	paralléliser le traitement des évènements de chacun des objets tout en
	maintenant une synchronisation globale.  La récupération des évènements en
	provenance du LAS est synchrone (effectuée à chaque \textit{step}).
	Cependant, la propagation des évènements des objets de la simulation vers
	les canaux de communication est asynchrone, par l'utilisation d'Observer,
	pour éviter la gestion du dispatching des évènements au simulateur.

	Les machines à états finis sont des secrets des différents objets de
	simulation et n'ont pas été explicités dans l'architecture.

		\subsection*{Justification des liens use du simulateur}

\begin{table}[!h]
\begin{tabularx}{\marginparsep+\marginparwidth+\marginparpush+\textwidth}{|l|l|X|}
Source & Destination & Justification 
Gui & Simulator & L'implémentation de l'interface pour lancer et stopper 
		  la simulation dépend de la spécfication externe de Simulator.

Gui & SimObjects & L'implémentation de la visualisation de la simulation dans
		   l'interface dépend de la spécification externe de SimObjects.
		   En effet, l'interface écoute les évènements générés par 
		   SimObjects définissant les objets à afficher.

Simulator & SimObjects & L'implémentation de la simulation dépend de l'interface
			 externe des SimObjects. Le simulateur ne pourra pas 
			 connecter les SimObjects entre eux si leur interface
			 change. 

Simulator & Map & L'implémentation de la simulation dépend de l'interface 
		  externe de Map car le Simulator ajoute et enlève les obstacles
		  de celle ci pour correspondre au scénario de simulation.

Simulator & Communication & L'implémentation de la simulation dépend de l'interface
			    externe du module de Communication. Si l'opération recieve
			    change sa spécification, le 
			    simulateur ne pourra pas foncitonner. 

Ambulance & Map		& L'implémentation de Ambulance dépend de l'interface externe
			de Map, En effet si l'opération nextPos de Map change de 
			spécification, l'ambulance ne se déplacera plus correctement.

Simulator & Scénario & L'implémentation du Simulator dépend des spécifications de
			Scénario car si l'opération nextStep de Scénario change de
			spécification, le simulateur ne simulera pas les bonnes
			étapes. 


\end{tabularx}
\end{table}

		
\chapter{Architecture physique}
	L'architecture physique des deux parties du logiciel développé sont
présenté ci-dessous.

\insertfiguremargin{phys-sys}{Architecture physique pour la partie du 
système.}

\insertfiguremargin{phys-simu}{Architecture physique pour la partie du 
simulateur.}


\chapter{Évènements à l'interface}
	Cette section décrit les différents messages qui sont échangés entre le
système logiciel et l'environement, simulé ou non. Ces évènement sont échangé
aux travers des différentes interfaces de communications possible entre 
les deux lieux : l'AVLS, le MDT et l'interface d'entrée des informations
pour les incidents.

\begin{description}
	\item [Incident:] L'événement incident est envoyé à partir du simulateur vers le système et comporte les informations suivantes: 		
	l'âge de la victime, enceinte ou non, la localisation de l'incident (typiquement l'adresse), et une description de l'incident.
	\item [AVLSMessage:]L'événement AVLSMessage est envoyé à partir du simulateur vers le système et comporte la posisition (coordonnées 					géographique) et l'identifiant d'un ambulance.
	\item [MDTMessage:] L'événement MDTMessage peut provenir soit du système soit du simulateur. Il y a plusieurs type de MDTMessage
\end{description}	


%MDTMessage du système vers le simulateur
\begin{table}[!h]
	\centering
	\begin{tabularx}{\marginparsep+\marginparwidth+\marginparpush+\textwidth}{|l|X|X|}
		\hline
		\textbf{nom du message} & \textbf{arguments} & \textbf{description} \\ \hline
		mobilisationOrder & incidentID, incidentPosition, ambulanceID & Le message de mobilisation est envoyé par le système à une ambulance (AmbulcanceID) afin de la mobiliser pour un incident (incidentID) qui à lieu à la position (incidentPosition) \\ \hline
		demobilisationOrder & incidentID, incidentPosition, ambulanceID & Le message demobilisationOrder est envoyé par le système à une ambulance (AmbulanceID) afin de la démobiliser pour l'incident (incidentId) se trouvant à la position (incidentPosition) \\ \hline
	\end{tabularx} 
	\caption{MDTMessage du système vers le simulateur}
	\label{tab:MDTMessage}
\end{table}

%MTDMessage du simulateur vers le système
\begin{table}[!h]
	\centering
	\begin{tabularx}{\marginparsep+\marginparwidth+\marginparpush+\textwidth}{|l|X|X|}
		\hline
		\textbf{nom du message} & \textbf{arguments du message} & \textbf{description du message} \\ \hline
		mobilisationConfirmation & incidentID, ambulanceID, un booleen yes/no & Un message de confirmation est envoyé par l'ambulance (ambulanceID) pour accepter (yes) ou refuser (no) l'ordre de mobilisation concernant l'incident (incidentID) \\ \hline 
		ambulanceBroken & ambulanceID & Message envoyé par l'ambulance (ambulanceID) pour signaler qu'elle est en panne \\ \hline
		ambulanceRepaired & ambulanceID & Message envoyé par l'ambulance (ambulanceID) pour signaler qu'elle est réparée \\ \hline
		obstacle & position & Un message Obstacle avec une position en argument est envoyé par une ambulance au système pour signaler qu'il existe une obstacle à cette position \\ \hline
		incidentCancelled & incidentID, ambulanceID & Un message incidentCancelled est envoyé par l'ambulance (ambulanceID) au système pour signaler que l'intervention pour l'incident (incidentID) est annulée \\ \hline
		incidentResolved & incidentID, ambulanceID & Un message incidentResolved est envoyé par l' ambulance (ambulanceID) au système pour signaler que l'intervention pour l'incident (incidentID) est résolue \\ \hline
		destinationUnreachable & ambulanceID, incidentID & Une message destinationUnreachable est envoyé par l'ambulance (ambulanceID) au système pour signaler que la position de l'incident (incidentID) est inaccessible \\ \hline
	\end{tabularx} 
	\caption{MDTMessage du simulateur vers le système}
	\label{tab:MDTMessage}
\end{table}


\chapter{Plan de développement}
	Cette partie présente le plan de développement de notre logiciel ainsi que
la répartition du travail au sein du groupe.

Le travail est réparti en équipe de développeurs, ces développeurs sont des
équipes de deux ou trois personnes au maximum, le but étant de minimiser les 
interactions tout en conservant une relecture et une liberté d'implémentation
aux équipes.

Dans chacune des phases, l'équipe se voit assigner un ensemble de module à 
écrire et à tester, sur base des tests black-box précédemment conçu. Ces tests
seront rédigé à l'aide JUnit et serviront également de tests de régressions.
Les tests seront écrits à l'aide de JUnit 4 et le code java sera écrit
en code compatible Java 6.

\section{Phase 1 : La gestion de l'information}
Cette phase permet de mettre en place tous les objets qui seront utilisé
par le reste de l'application. Ces objets seront stocké, dans un premier
temps, pour la durée de l'exécution du logiciel.

Les modules concernés par cette phase sont, pour la
partie système: Incident, Map, Ambulance et pour la partie simulateur : SimObjects, Map.
La répartition du travail pour 
les différents module et pour les équipes est décrite ci-dessous:

\noindent\begin{tabularx}{\textwidth}{|l|l|X|}
\hline
Module			& 	Équipe de développement & Équipe de test\\
\hline
Incident	&	Team A 	&	Team B	\\
Map 		&	Team B	&	Team A	\\
Ambulance	&	Team A  &	Team B	\\
SimObjects	&	Team A  &	Team B	\\
Map			&	Team B  &	Team A	\\
\hline
\end{tabularx}

Fin de la première phase: 27 novembre.

\section{Phase 2 : Communication}
Cette phase met en place la communication entre les deux mondes.

Les modules concernés par cette seconde phase sont les suivants, pour
le système : Communicator, Broker et pour le simulateur : Communication, CallSimul, AVLS et MDT.
À nouveau, la répartition est présentée dans le tableau suivant:

\noindent\begin{tabularx}{\textwidth}{|l|l|X|}
\hline
Module			& 	Équipe de développement & Équipe de test\\
\hline
Communicator	&	Team B 	&	Team A\\
Broker			&	Team B	&	Team A\\
Communication 	&	Team B	&	Team A	\\
CallSimul		&	Team A	&	Team B	\\
AVLS			&	Team A	&	Team B	\\
MDT				&	Team A	&	Team B	\\
\hline
\end{tabularx}

Fin de la seconde phase: 2 décembre.

\section{Phase 3 : Le coeur}
Cette phase va s'appuyer sur les phases précédentes afin de les exploiter
et de faire en sorte que le logiciel fasse ce pourquoi ce dernier a été conçu.

Les modules concernés sont les suivants: pour le système: IncidentProcessor,
AmbulanceChooser, Mobilizer et Resolver, pour le simulateur: Simulator,
Scenario et FileScenario.
À nouveau, la répartition est présentée dans le tableau suivant:

\noindent\begin{tabularx}{\textwidth}{|l|l|X|}
\hline
Module			& 	Équipe de développement & Équipe de test\\
\hline
IncidentProcessor	&	Team A 	&	Team B\\
AmbulanceChooser	&	Team B	&	Team A\\
Mobilizer 			&	Team A	&	Team B	\\
Resolver			&	Team A	&	Team B	\\
Simulator			&	Team B	&	Team A	\\
Scenario			&	Team A	&	Team B	\\
FileScenario		&	Team B	&	Team A	\\
\hline
\end{tabularx}

Fin de la seconde phase: 11 décembre.

\section{Phase 4 : L'enrichissement}
Cette phase est l'occasion d'ajouter des modules à notre architecture afin de 
proposer une plus grand nombre de fonctionnalité.

En fonction du temps et des affinités des équipes, il sera possible d'implémenter
certaines de ces fonctionnalités.

Parmis ces fonctionnalités, nous avons:
\begin{itemize}
	\item Utilisation d'une base de donnée relationnelle pour sauvegarder les 
		  données de manière pérenne.
	\item L'ajout d'un module de statistique du coté du simulateur
	\item L'ajout d'un module de statistique du coté du système
	\item L'ajout d'un module s'occupant du placement des ambulances
		  afin de maximiser la couverture géographique
	\item L'utilisation d'une carte plus complexe, pour louvain-la-neuve par
		  exemple.
	\item Le développement de canaux de communication supplémentaire (Radio, 
	      téléphonne, etc.)
	\item L'ajout de scénario probabiliste
	\item L'ajout de scénario généré manuellement
	\item ...
\end{itemize}


\end{document}
